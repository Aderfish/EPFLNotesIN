\documentclass[a4paper]{article}

% Expanded on 2021-12-06 at 13:14:43.

\usepackage{../../style}

\title{Algèbre linéaire}
\author{Joachim Favre}
\date{Lundi 06 décembre 2021}

\begin{document}
\maketitle

\lecture{22}{2021-12-06}{La la la Schtroumpf la la}{
    \begin{itemize}[left=0pt]
        \item Explication de l'algorithme du Grand Schtroumpf (euh de Gram-Schmidt pardon). 
        \item Explication de la méthode pour factoriser une matrice dans la forme $QR$. 
        \item Étude du cas de la factorisation $QR$ matrice carrée, et meilleure compréhension du déterminant d'une matrice.
    \end{itemize}
}

\subsection{Orthonormalisation}
\parag{Introduction}{
    Soit $W = \vect\left\{\bvec{x}_1, \bvec{x}_2\right\}$, où: 
    \[\bvec{x}_1 = \begin{bmatrix} 3 \\ 6 \\ 0 \end{bmatrix}, \mathspace \bvec{x} = \begin{bmatrix} 1 \\ 2 \\ 2 \end{bmatrix} \]
    
    Pour calculer des projections orthogonales sur $W$, ce serait pratique d'avoir une base orthonormée. La base $\mathcal{B} = \left(\bvec{x}_1, \bvec{x}_2\right)$ n'est même pas orthogonale: 
    \[\bvec{x}_1 \dotprod \bvec{x}_1 = 45, \mathspace \bvec{x}_1 \dotprod \bvec{x}_2 = 15, \mathspace \bvec{x}_2 \dotprod \bvec{x}_2 = 9\]
    
    On va donc \important{orthonormaliser} cette base.

    \imagehere[0.7]{MethodeOrthonormalisation.png}

    Pour commencer, normalisons $\bvec{x}_1$: 
    \[\left\|\bvec{x}_1\right\|^2 = 3^2 + 6^2 + 0^2 = 45 \implies \bvec{u}_1 = \frac{1}{\left\|\bvec{x}_1\right\|} \bvec{x}_1 = \frac{1}{3\sqrt{5}} \begin{bmatrix} 3 \\ 6 \\ 0 \end{bmatrix} = \frac{1}{\sqrt{5}} \begin{bmatrix} 1 \\ 2 \\ 0 \end{bmatrix} \]

    Maintenant, construisons $\bvec{v}_2$. On projette $\bvec{x}_2$ sur la droite donnée par $W_1$, donc la droite engendrée par $\bvec{u}_1$, puis on va soustraire ce résultat à $\bvec{x}_2$:
    \[\bvec{v}_2 = \bvec{x}_2 - \proj_{W_1}\left(\bvec{x}_2\right) = \bvec{x}_2 - \left(\bvec{x}_2 \dotprod \bvec{u}_1\right) \bvec{u}_1 = \begin{bmatrix} 1 \\ 2 \\ 2 \end{bmatrix} - \frac{5}{\sqrt{5}} \cdot \frac{1}{\sqrt{5}} \begin{bmatrix} 1 \\ 2 \\ 0 \end{bmatrix} = \begin{bmatrix} 0 \\ 0 \\ 2 \end{bmatrix} \]
    
    Finalement, nous pouvons normaliser $\bvec{v}_2$ pour obtenir le deuxième vecteur de notre base orthonormée: 
    \[\bvec{u}_2 = \frac{1}{\left\|\bvec{v}\right\|} \bvec{v} = \frac{1}{2} \begin{bmatrix} 0 \\ 0 \\ 2 \end{bmatrix} = \begin{bmatrix} 0 \\ 0 \\ 1 \end{bmatrix} \]
    
    Ainsi, par construction, $\mathcal{U} = \left(\bvec{u}_1, \bvec{u}_2\right)$ est une base orthonormée, et $\vect\left\{\bvec{u}_1, \bvec{u}_2\right\} = \vect\left\{\bvec{x}_1, \bvec{x}_2\right\}$

    \subparag{Observation 1}{
        On remarque que: 
        \[\bvec{x}_1 = \left\|\bvec{x}_1\right\|\bvec{u}_1\]

        De plus, en manipulant un petit peu les équations ci-dessus: 
        \[\bvec{x}_2 = \left(\bvec{x}_2 \dotprod \bvec{u}_1\right)\bvec{u}_1 + \bvec{v}_2 = \left(\bvec{x}_2 \dotprod \bvec{u}_1\right) \bvec{u}_1 + \left\|\bvec{v}_2\right\| \bvec{u}_2\]
        
        Ainsi, on peut écrire ceci sous la forme d'un produit matriciel: 
        \[\begin{bmatrix}  &  \\ \bvec{x}_1 & \bvec{x}_2 \\  &  \end{bmatrix} = \begin{bmatrix}  &  \\ \bvec{u}_1 & \bvec{u}_2 \\  &  \end{bmatrix} \begin{bmatrix} \left\|\bvec{x}_1\right\| & \bvec{x}_2 \dotprod \bvec{u}_1 \\ 0 & \left\|\bvec{v}_2\right\| \end{bmatrix} \implies A = QR\]
        où $A$ set la matrice dont les colonnes forment une base quelconque de $W$, $Q$ est une matrice dont les colonnes forment une base orthonormée de $W$, et $R$ est une matrice triangulaire supérieure. On se rendra compte qu'on pourra toujours factoriser $A$ sous cette forme, si les colonnes de $A$ sont linéairement indépendants.
    }

    \subparag{Observation 2}{
        Non seulement $\left(\bvec{u}_1, \bvec{u}_2\right)$ est une base orthonormée de $\vect\left\{\bvec{x}_1, \bvec{x}_2\right\}$, mais aussi $\left(\bvec{u}_1\right)$ est une base orthonormée de $\vect\left\{\bvec{x}_1\right\}$.
    }

    \subparag{Remarque}{
        Cet algorithme s'appelle formellement l'algorithme de Gram-Schmidt. Cependant, il est beaucoup plus intéressant de lui donner des noms amicaux. Un très bon ami, un \textit{Génie en la Physique}, l'appelle ``l'algorithme du Grand Schmit'', ou ``l'algorithme du Grand Schtroumpf''.
    }
}

\parag{Avec 3 vecteurs}{
    Soient les trois vecteurs suivants: 
    \[\bvec{x}_1 = \begin{bmatrix} 1 \\ 1 \\ 1 \\ 1 \end{bmatrix}, \mathspace \bvec{x}_2 = \begin{bmatrix} 0 \\ 1 \\ 1 \\ 1 \end{bmatrix}, \mathspace \bvec{x}_3 = \begin{bmatrix} 0 \\ 0 \\ 1 \\ 1 \end{bmatrix} \]
    
    On peut se rendre compte que ces trois vecteurs sont linéairement indépendants, donc ils forment une base pour $W = \vect\left\{\bvec{x}_1, \bvec{x}_2, \bvec{x}_3\right\}$. Comme la dernière fois, on veut construire une base orthonormée $\mathcal{U} = \left(\bvec{u}_1, \bvec{u}_2, \bvec{u}_3\right)$.

    Pour commencer, calculons $\bvec{u}_1$ en normalisant $\bvec{x}_1$: 
    \[\left\|\bvec{x}_1\right\|^2 = 4 \implies \bvec{u}_1 = \frac{1}{\left\|\bvec{x}_1\right\|} \bvec{x}_1 = \frac{1}{2} \begin{bmatrix} 1 \\ 1 \\ 1 \\ 1 \end{bmatrix} \]
    
    Maintenant, pour calculer $\bvec{v}_2$, on retranche la projection de $\bvec{x}_2$ sur $W_1 = \vect\left\{\bvec{u}_1\right\}$ à $\bvec{x}_2$: 
    \[\bvec{v}_2 = \bvec{x}_2 - \proj_{W_1}\left(\bvec{x}_2\right) = \bvec{x}_2 - \left(\bvec{x}_2 \dotprod \bvec{u}_1\right)\bvec{u}_1 = \begin{bmatrix} 0 \\ 1 \\ 1 \\ 1 \end{bmatrix} - \frac{3}{2} \cdot \frac{1}{2}\begin{bmatrix} 1 \\ 1 \\ 1 \\ 1 \end{bmatrix} = \begin{bmatrix} -3 / 4 \\ 1 / 4 \\ 1 / 4 \\ 1 / 4 \end{bmatrix} = \frac{1}{4} \begin{bmatrix} -3 \\ 1 \\ 1 \\ 1 \end{bmatrix} \]
    
    Ainsi, en normalisant $\bvec{v_2}$, on obtient $\bvec{u}_1$: 
    \[\bvec{u}_2 = \frac{1}{\bvec{v}_2} \bvec{v}_2 = \frac{2}{\sqrt{3}} \frac{1}{4} \begin{bmatrix} -3 \\ 1 \\ 1 \\ 1 \end{bmatrix} = \frac{1}{2\sqrt{3}} \begin{bmatrix} -3 \\ 1 \\ 1 \\ 1 \end{bmatrix} \]

    On remarque que, jusque là, le fait qu'on ait 3 vecteurs ne change rien : on n'a jamais utilisé le troisième vecteur.

    \imagehere[0.5]{MethodeOrthonormalisation3Vecteurs.png}

    On va faire le même raisonnement que pour les dernières fois : on va projeter $\bvec{x}_3$ sur $W_2 = \vect\left\{\bvec{u}_1, \bvec{u}_2\right\}$, puis retrancher ce résultat à $\bvec{x}_3$ pour obtenir $\bvec{v_3}$. La projection est calculable facilement, puisque $\left(\bvec{u}_1, \bvec{u}_2\right)$ est une base orthogonale.
    
    \[\bvec{v}_3 = \bvec{x}_3 - \proj_{W_2}\left(\bvec{x}_3\right) = \bvec{x}_3 - \left(\bvec{x_3} \dotprod \bvec{u}_1\right)\bvec{u}_1 - \left(\bvec{x}_3 \dotprod \bvec{u}_2\right)\bvec{u}_2\]

    Ainsi, on obtient 
    \[\bvec{v}_3 = \begin{bmatrix} 0 \\ 0 \\ 1 \\ 1 \end{bmatrix} - 1\cdot \frac{1}{2} \begin{bmatrix} 1 \\ 1 \\ 1 \\ 1 \end{bmatrix} - \frac{1}{\sqrt{3}} \frac{1}{2\sqrt{3}} \begin{bmatrix} -3 \\ 1 \\ 1 \\ 1 \end{bmatrix} = \begin{bmatrix} 0 \\ 0 \\ 1 \\ 1 \end{bmatrix} - \begin{bmatrix} 1 / 2 \\ 1 / 2 \\ 1 / 2 \\ 1 / 2 \end{bmatrix}  - \begin{bmatrix} -1 / 6 \\ 1 / 6 \\ 1 / 6 \\ 1 / 6 \end{bmatrix} = \begin{bmatrix} 0 \\ -2 / 3 \\ 1 / 3 \\ 1 / 3 \end{bmatrix} = \frac{1}{3} \begin{bmatrix} 0 \\ -2 \\ 1 \\ 1 \end{bmatrix} \]
    
    Finalement, on peut normaliser $\bvec{v}_3$ pour obtenir $\bvec{u}_3$: 
    \[\bvec{u}_3 = \frac{1}{\left\|\bvec{v}_3\right\|} \bvec{v}_3 = \frac{\sqrt{3}}{\sqrt{2}} \cdot \frac{1}{3} \begin{bmatrix} 0 \\ -2 \\ 1 \\ 1 \end{bmatrix} = \frac{1}{\sqrt{6}} \begin{bmatrix} 0 \\ -2 \\ 1 \\ 1 \end{bmatrix} \]

    On a donc construit la base orthonormée $U = \left(\bvec{u}_1, \bvec{u}_2, \bvec{u}_3\right)$ pour $W = \vect\left\{\bvec{x}_1, \bvec{x}_2, \bvec{x}_3\right\}$, où: 
    \[\bvec{u}_1 = \frac{1}{2} \begin{bmatrix} 1 \\ 1 \\ 1 \\ 1 \end{bmatrix}, \mathspace \bvec{u}_2 = \frac{1}{2\sqrt{3}} \begin{bmatrix} -3 \\ 1 \\ 1 \\ 1 \end{bmatrix}, \mathspace \bvec{u}_3 = \frac{1}{\sqrt{6}} \begin{bmatrix} 0 \\ -2 \\ 1 \\ 1 \end{bmatrix}  \]

    \subparag{Observation}{
        On remarque que:
        \begin{itemize}
            \item $\left\{\bvec{u}_1\right\}$ est une base orthonormée pour $W = \vect\left\{\bvec{x}_1\right\}$.
            \item $\left\{\bvec{u}_1, \bvec{u}_2\right\}$ est une base orthonormée pour $W = \vect\left\{\bvec{x}_1, \bvec{x}_2\right\}$.
            \item $\left\{\bvec{u}_1, \bvec{u}_2, \bvec{u}_3\right\}$ est une base orthonormée pour $W = \vect\left\{\bvec{x}_1, \bvec{x}_{2}, \bvec{x}_3\right\}$.
        \end{itemize}
        

        De plus, on voit aussi que: 
        \[\bvec{x}_1 = \left\|\bvec{x}_1\right\| \bvec{u}_1\]
        \[\bvec{x}_2 = \left(\bvec{x}_2 \dotprod \bvec{u}_1\right)\bvec{u}_1 + \bvec{v}_2 = \left(\bvec{x}_2 \dotprod \bvec{u}_1\right) \bvec{u}_1 + \left\|\bvec{v}_2\right\|\bvec{u}_2\]
        \begin{multiequality}
        \bvec{x}_3 =\ & \left(\bvec{x}_3 \dotprod \bvec{u}_1\right)\bvec{u}_1 + \left(\bvec{x}_3 \dotprod \bvec{u}_2\right) \bvec{u}_2 + \bvec{v}_3 \\
        =\ & \left(\bvec{x}_3 \dotprod \bvec{u}_1\right)\bvec{u}_1 + \left(\bvec{x}_3 \dotprod \bvec{u}_3\right) \bvec{u}_2 + \left\|\bvec{v}_3\right\| \bvec{u}_3
        \end{multiequality}
        
        On peut donc à nouveau écrire $A = QR$: 
        \[\begin{bmatrix}  &  &  \\ \bvec{x}_1 & \bvec{x}_2 & \bvec{x}_3 \\  &  &  \end{bmatrix} = \begin{bmatrix}  &  &  \\ \bvec{u}_1 & \bvec{u}_2 & \bvec{u}_3 \\  &  &  \end{bmatrix} \begin{bmatrix} \left\|\bvec{x}_1\right\| & \bvec{x}_2 \dotprod \bvec{u}_1 & \bvec{x}_3 \dotprod \bvec{u}_1 \\ 0 & \left\|\bvec{v}_2\right\| & \bvec{x}_3 \dotprod \bvec{u}_2 \\ 0 & 0 & \left\|\bvec{v}_3\right\| \end{bmatrix} \]
        
        Ainsi, puisque les colonnes de $Q$ forment une base orthonormée pour $\im A$, on a: 
        \[Q^T Q = I_3\]
        \[\proj_{\im A}\left(\bvec{y}\right) = QQ^T \bvec{y}\]

        Le plus simple pour trouver la matrice qui projette sur l'espace image de $A$ est donc d'appliquer le Grand Schtroumpf (Gram-Schmidt), de factoriser $A$ sous la forme $QR$, puis de calculer $QQ^T$
    }
}

\parag{Dépendance linéaire}{
    Nous pouvons nous rendre compte que, si nous appliquons le Grand Schtroumpf sur des vecteurs qui sont linéairement dépendants, alors on aurait obtenu le vecteur $\bvec{0}$ à un endroit. Cependant, puisqu'en suite nous devons diviser ce vecteur par sa norme, ce qui est une division par 0, l'algorithme casse bel et bien.

    En effet, la projection d'un vecteur dans un espace auquel il appartient est égal à ce vecteur. Ainsi, quand on va retrancher ce résultat à ce vecteur, alors cela va nous donner $\bvec{0}$.

    Ceci est cohérent puisque notre but est d'orthonormaliser une \textit{base}, ainsi si les vecteurs n'étaient pas linéairement indépendants, alors ce n'était pas une base au départ.

}

\parag{Gram-Schmidt et factorisation QR}{
    Soit $A$ une matrice de taille $m \times n$ dont les colonnes sont linéairement indépendants. En d'autres mots, $\left(\bvec{x}_1, \ldots, \bvec{x}_n\right)$ est une base de $W = \vect\left\{\bvec{x}_1, \ldots, \bvec{x}_n\right\} = \im A$.

    \important{L'algorithme de Gram-Schmidt} (Grand Schtroumpf) revient à calculer: 
    \[\bvec{v}_n = \bvec{x}_n - \proj_{W_{n-1}}\left(\bvec{x}_n\right) = \bvec{x}_n - \left(\bvec{x}_n \dotprod \bvec{u}_1\right) \bvec{u}_1 - \ldots - \left(\bvec{x}_n \dotprod \bvec{u}_{n-1}\right)\bvec{u}_{n-1}\]
    \[\bvec{u}_n = \frac{1}{\left\|\bvec{v}_n\right\|} \bvec{v}_n\]
    où $W_{n-1} = \vect\left\{\bvec{x}_1, \ldots, \bvec{x}_{n-1}\right\}$.

    Ces calculs nous donnent $\left(\bvec{u}_1, \ldots, \bvec{u}_n\right)$, une \important{base orthonormée} de $W = \im A$. De plus, ils révèlent la factorisation $QR$ de $A$: 
    \[A = QR, \mathspace Q = \begin{bmatrix}  &  &  \\ \bvec{u}_1 & \ldots & \bvec{u}_n \\  &  &  \end{bmatrix}, \mathspace R = \begin{bmatrix} \left\|\bvec{x}_1\right\| & \bvec{x}_2 \dotprod \bvec{u}_1 & \ldots & \bvec{x}_n \dotprod \bvec{u}_1 \\ 0 & \left\|\bvec{v}_2\right\| & \ldots & \bvec{x}_n \dotprod \bvec{u}_2 \\ \vdots & \vdots & \ddots & \vdots \\ 0 & 0 & \ldots & \left\|\bvec{v}_n\right\| \end{bmatrix} \]
}

\parag{Matrice carrée}{
    Étudions la factorisation $QR$ d'une matrice $A \in \mathbb{R}^{n \times n}$ carrée. 

    Appliquons le Grand Schtroumpf à ses colonnes. Il y a deux cas possibles. Si cet algorithme échoue (il y a une division par zéro), alors $A$ n'est pas inversible. S'il aboutit, alors on trouve $Q \in \mathbb{R}^{n \times n}$ et $R \in \mathbb{R}^{n \times n}$ telles que: 
    \[A = QR, \mathspace Q^T Q = Q Q^T = I_n \implies Q^{-1} = Q^T\]
    et où $R$ est triangulaire est inversible (elle n'a que des nombres strictement positifs sur sa diagonale, donc son déterminant n'est pas nul).

    On peut utiliser ce résultat pour résoudre $A \bvec{x} = \bvec{b}$. En effet, on remarque que:
    \[A \bvec{x} = \bvec{b} \iff QR \bvec{x} = \bvec{b} \iff Q^T Q R \bvec{x} = Q^T \bvec{b} \iff R \bvec{x} = Q^T \bvec{b}\]
    
    Or, $Q^T \bvec{b}$ est facilement calculable, et, puisque $R$ est triangulaire, il est très facile de résoudre ce système.

    De plus, on peut aussi utiliser ce résultat pour calculer (et mieux comprendre) $\det\left(A\right)$: 
    \[1 = \det\left(I\right) = \det\left(Q^T Q\right) = \det\left(Q^T\right) \det\left(Q\right) = \det\left(Q\right)^2 \implies \det\left(Q\right) = \pm 1\]
    
    Ainsi, on peut utiliser ceci pour calculer $\det\left(A\right)$: 
    \[\det\left(A\right) = \det\left(QR\right) = \det\left(Q\right)\det\left(R\right) = \pm\det\left(R\right) \implies \left|\det\left(A\right)\right| = \left|\det\left(R\right)\right|\]
    
    Cependant, on sait que $R$ est une matrice triangulaire supérieure, ainsi il est très simple de calculer son déterminant: c'est le produit de ses entrées diagonales (mais les calculs pour y arriver rendent cette méthode plus compliquée que celles que nous avions vues jusque là). Ainsi, on en déduit que: 
    \[\left|\det\left(A\right)\right| = \left\|\bvec{x}_1\right\| \left\|\bvec{v}_2\right\| \left\|\bvec{v}_3\right\|\cdots \left\|\bvec{v}_n\right\|\]
    
    \imagehere[0.5]{DeterminantAireParallelogramme.png}

    On peut voir que l'aire du parallélogramme défini par $\bvec{x}_1, \bvec{x}_n$ est égale à l'aire du rectangle définie par $\bvec{x}_1$ et $\bvec{v}_2$, qui vaut $\left\|\bvec{x}_1\right\| \cdot \left\|\bvec{v}_2\right\| = \det\left(R\right)$.

    Ceci est vrai pour tout $n$, donc $\left|\det\left(A\right)\right|$ est l'hyper-volume de l'hyper-parallélépipède de dimension $n$

}



\end{document}
