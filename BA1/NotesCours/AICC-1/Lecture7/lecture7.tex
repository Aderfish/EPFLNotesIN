\documentclass[a4paper]{article}

% Expanded on 2021-10-12 at 08:39:21.

\usepackage{../../style}

\title{AICC}
\author{Joachim Favre}
\date{Mardi 12 octobre 2021}

\begin{document}
\maketitle

\lecture{7}{2021-10-12}{The end of sets and the beginning of functions}{
\begin{itemize}[left=0pt]
    \item Definition of set equality and subsets.
    \item Definition of power sets, tuples, Cartesian product, truth sets of predicates and set cardinality.
    \item Definition of the set operations, i.e. union (and its cardinality), intersection, difference, complement and symmetric difference.
    \item Explanation on how to prove set equalities.
    \item Definition of functions and explanation of some of its terminology.
    \item Definition of injective (one-to-one), surjective (onto) and bijective functions.
\end{itemize}

}

\parag{Definition: Set equality}{
    Two sets $A$ and $B$ are equal, written $A = B$, if and only if $A$ and $B$ have the same elements. In other words:
    \[A = B \equiv \forall x\left(x \in A \leftrightarrow x \in B\right)\]
}

\parag{Definition: Subsets}{
    The set $A$ is as \important{subset} of $B$, written $A \subseteq B$, if and only if every element of $A$ is also an element of $B$. In other words:
    \[A \subseteq B \equiv \forall x\left(x \in A \to x \in B\right)\]

    We say that $A$ is a \important{proper subset} of $B$, written $A \subset B$, if and only if $A \subseteq B$ but $A \neq B$; i.e:
    \[\forall x\left(x \in A \to x \in B\right) \land \exists x\left(x \in B \land x \not\in A\right)\]

    \subparag{With important sets}{
        Note that, for any set $S$:
        \[\o \subseteq S, \mathspace S \subseteq S\]
    }

    \subparag{Showing this property}{
        To show that $A$ is a subset of $B$, we need to show that if $x$ belongs to $A$, then $x$ also belongs to $B$. To show that $A$ is not a subset of $B$, we need to find an element $x$ such that $x \in A$ but $x \not\in B$. To show that $A$ is a proper subset of $B$, we need to show that $a$ is a subset of $B$, and find an element $x$ such that $x \in B$ and $x \not\in A$.
    }

}

\parag{Showing equality}{
    We can note that:
    \[A = B \equiv \forall x\left(x \in A \leftrightarrow x \in B\right) \equiv \forall x\left(x \in A \to x \in B \land x \in B \to x \in A\right) \equiv A \subseteq B \land B \subseteq A\]

    So, proving that two sets consists in proving that one set is included in the other, and vice-versa.
}

\parag{Venn Diagrams}{
    Venn Diagrams are pictures of sets, drawn as subsets of some universal set $U$. They are very useful for understanding, but can never be used for proofs.

    \imagehere{VennDiagrams.png}
}


\subsection{Constructing sets}

\parag{Power sets}{
    The set of all subsets of a set $A$, written $\mathcal{P}\left(A\right)$, is called the \important{power set} of $A$.

    \subparag{Example}{
        If $A = \left\{a, b\right\}$, then:
        \[\mathcal{P}\left(A\right) = \left\{\o, \left\{a\right\}, \left\{b\right\}, \left\{a, b\right\}\right\}\]
    }

    \subparag{Remark}{
        Note that, for any set $A$, then:;
        \[\o \subseteq \mathcal{P}\left(A\right) \mathspace \text{ and } \mathspace A \subseteq \mathcal{P}\left(A\right)\]
    }
}

\parag{Tuples}{
    The \important{ordered $n$-tuples} $\left(a_1, \ldots, a_n\right)$ is the ordered collections that has $a_1$ as its first element, $a_2$ as its second elements, and so on until $a_n$, which is its last element.

    2-tuples are called \important{ordered pairs}.

    \subparag{Remarks}{
        We can have multiple times the same element in a tuple. For example:
        \[\left(1, 1, 1\right) \neq \left(1, 1\right) \neq \left(1\right)\]

        Concerning ordered pairs, the ``ordered'' is important. Indeed, in general:
        \[\left(a, b\right) \neq \left(b, a\right)\]
    }

    \subparag{Equality}{
        Two $n$-tuples are equal if and only if their corresponding elements are equal:
        \[\left(a_1, \ldots, a_n\right) = \left(b_1, \ldots, b_n\right) \iff a_1 = b_1 \land \ldots \land a_n = b_n\]
    }
}

\parag{Cartesian product}{
    The \important{Cartesian product} of two sets $A$ and $B$, denoted by $A \times B$, is the set of ordered pairs $\left(a, b\right)$ where $a \in A$ and $b \in B$:
    \[A \times B = \left\{\left(a, b\right) | a \in A \land b \in B\right\}\]

    A subset $R$ of the Cartesian product $A \times B$ is called a \important{relation} from the set $A$ to the set $B$.

    The \important{Cartesian products} of the sets $A_1, \ldots, A_n$, denoted $A_1 \times \ldots \times A_n$ is the set of the ordered $n$-tuples $\left(a_1, \ldots, a_n\right)$, where $a_i$ belongs to $A_i$:
    \[A_1 \times \ldots \times A_n = \left\{\left(a_1, \ldots, a_n\right) | a_i \in A_i\right\}\]

    \subparag{Remark}{
        We can note that, in general, it is not commutative:
        \[A \times B \neq B \times A\]

        Note that, in general, it is not associative either:
        \[A \times \left(B \times C\right) \neq \left(A \times B\right) \times C\]
    }

    \subparag{Example}{
        Let $A = \left\{a, b\right\}$ and $B = \left\{1, 2, 3\right\}$. Then:
        \[A \times B = \left\{\left(a, 1\right), \left(a,2\right), \left(a,3\right), \left(b, 1\right), \left(b,2\right), \left(b,3\right)\right\}\]

        A relation is any subset, so, for example:
        \[R = \left\{\left(a,2\right), \left(b,1\right)\right\}\]

        Moreover, if $A = \left\{0, 1\right\}$, $B = \left\{1, 2\right\}$ and $C = \left\{0, 1, 2\right\}$, then:
        \[A \times B \times C = \left\{\left(0, 1,0\right), \left(0, 1, 1\right), \left(0, 1, 2\right), \left(0, 2, 0\right), \ldots, \left(1, 2, 2\right)\right\}\]
    }
}

\parag{Truth set}{
    Given a predicate $P$ and a domain $D$, we define the \important{truth set} of $P$ to be the set of elements in $D$ for which $P\left(x\right)$ is true. This is denoted by:
    \[\left\{x \in D | P\left(x\right)\right\}\]

    \subparag{Example}{
        The truth set of $P\left(x\right)$ where $P\left(x\right) := \left|x\right| = 1$ and the domain is the integers, is the set $\left\{-1, 1\right\}$.
    }
}

\parag{Set cardinality}{
    If there are $n \in \mathbb{N}$ distinct elements in a set $S$, we say that $S$ is \important{finite} (which basically mean that we could write the set down), otherwise it is \important{infinite}.

    The \important{cardinality} of a finite set $S$, denoted $\left|S\right|$, is the number of (distinct) elements of $S$.

    \subparag{Examples}{
        We see the following properties:
        \[\left|A\right| = n \implies \left|\mathcal{P}\left(A\right)\right| = 2^n\]
        \[\left|A\right| = n, \left|B\right| = m \implies \left|A \times B\right| = n\cdot m\]
        \[\left|\o\right| = 0, \mathspace \left|\left\{\o\right\}\right| = 1\]

        Finally, for example, the set of integers is infinite.
    }
}

\parag{Example}{
    The power set of the empty set can be sometimes a bit tricky. For example:
    \begin{itemize}
        \item $\mathcal{P}\left(\o\right) = \left\{\o\right\}$
        \item $\mathcal{P}\left(\mathcal{P}\left(\o\right)\right) = \mathcal{P}\left(\left\{\o\right\}\right) = \left\{\o, \left\{\o\right\}\right\}$
        \item $\mathcal{P}\left(\mathcal{P}\left(\left\{\o\right\}\right)\right) = \left\{\o, \left\{\o\right\}, \left\{\left\{\o\right\}\right\}, \left\{\o, \left\{\o\right\}\right\}\right\}$
    \end{itemize}

    We can make the following conjecture:
    \[\left|\mathcal{P}^n \left(\o\right)\right| = 2^{n}\]

    Where $P^n\left(A\right)$ is the repeated power set over the set $A$.
}

\parag{Proposition}{
    If $\mathcal{P}\left(A\right) = \mathcal{P}\left(B\right)$, then $A = B$.

    \subparag{Proof}{
        Let's do a proof by contraposition. Since $A \neq B$, then there exists an element $x$ which is in a set but not in the other. Let's say wlog that $x \in A$ and $x \not \in B$.

        So:
        \[\left\{x\right\} \subset \mathcal{P}\left(A\right), \mathspace \left\{x\right\} \not \in \mathcal{P}\left(B\right)\]

        Thus, $\mathcal{P}\left(A\right) \neq \mathcal{P}\left(B\right)$.

        \qed
    }
}

\subsection{Set operations}

\parag{Set operations}{
    \begin{center}
    \begin{tabularx}{\linewidth}{|>{\hsize=0.15\hsize}C>{\hsize=0.25\hsize}C>{\hsize=0.38\hsize}C>{\hsize=0.22\hsize}X|}
        \hline
        \fullbf{Name} & \fullbf{Definition} & \fullbf{Example} & \fullbf{Venn \mbox{Diagram}}  \\
        \hline
        \textit{Union} & $\left\{x | x \in A \lor x \in B\right\}$ & $\left\{1, 2\right\} \cup \left\{2, 3\right\} = \left\{1, 2, 3\right\}$ & \imagehere[0.2]{SetUnion.png} \\
        \hline
        \textit{Intersection} & $\left\{x | x \in A \land x \in B\right\}$ & $\left\{1, 2\right\} \cap \left\{2, 3\right\} = \left\{2\right\}$ & \imagehere[0.2]{SetIntersection.png}  \\
        \hline
        \textit{Difference} & $\left\{x | x \in A \land x \not\in B\right\}$ & $\left\{1, 2, 3\right\} \setminus \left\{2, 4\right\} = \left\{1, 2\right\}$& \imagehere[0.2]{SetDifference.png}  \\
        \hline
        \textit{Complement} & $\left\{x \in U | x \not \in A\right\}$ & $\bar{\left(-\infty, 3\right]} = \left(3, +\infty\right)$ & \imagehere[0.2]{SetComplement.png}  \\
        \hline
        \textit{Symmetric Difference} & $\left(A \setminus B\right) \cup \left(B \setminus A\right)$ & $\left\{1, 2, 3\right\} \oplus \left\{2, 3, 4\right\} = \left\{1, 4\right\}$ & \imagehere[0.2]{SymmetricDifference.png}  \\
        \hline
    \end{tabularx}
    \end{center}

    \begin{center}
    \begin{tabularx}{\linewidth}{|>{\hsize=0.2\hsize}C>{\hsize=0.3\hsize}C>{\hsize=0.5\hsize}X|}
        \hline
        \fullbf{Name} & \fullbf{Analogy with propositional connectives} & \fullbf{Comment} \\
        \hline
        \textit{Union} & $p \lor q$ & \\
        \hline
        \textit{Intersection} & $p \land q$ & If $A \cap B = \o$ then the sets are said to be \important{disjoint}. \\
        \hline
        \textit{Difference} &  & We can see the following property:
        \[A \setminus B = A \cap \bar{B}\]
        The difference of $A$ and $B$ is also called the complement of $B$ with respect to $A$  \\
        \hline
        \textit{Complement} & $\lnot p$ & We can see the following property:
        \[\bar{A} = U \setminus A\]
        The complement of $A$ is also denoted by $A^C$. \\
        \hline
        \textit{Symmetric \mbox{Difference}} & $p \oplus q$ &  \\
        \hline
    \end{tabularx}
    \end{center}

}

\parag{Cardinality of set Union}{
    The principle of Inclusion-Exclusion tells us that:
    \[\left|A \cup B\right| = \left|A\right| + \left|B\right| - \left|A \cap B\right|\]

    Indeed, we would count twice $\left|A \cap B \right|$ else. This result will be very useful later in the course, in counting.
}


\subsection{Set identities}
\parag{Set identities}{
    Set identities can be understood as analogues of logical equivalences in propositional logic.

    For example, we have De Morgan Laws for sets:
    \[\bar{A \cap B} = \bar{A} \cup \bar{B}, \mathspace \bar{A \cup B} = \bar{A} \cap \bar{B}\]

    To prove such identities, we have different ways. The first one is to use set-builder notation and propositional logic. The second one is to prove that each set is a subset of the other (as mentioned earlier). The third one is to do a membership table (this is an analogue of truth tables).
}

\parag{Example: set-builder notation}{
    Let's use the set-builder notation to prove that:
    \[\bar{A \cap B} = \bar{A} \cup \bar{B}\]

    This can be done using the following steps:
    \begin{multiequality}
    \bar{A \cap B} & = \left\{x | x \not\in A \cap B\right\} \\
    & = \left\{x | \lnot x \in A \cap B \right\} \\
    & = \left\{x | \lnot\left(x \in A \land x \in B\right)\right\} \\
    & = \left\{x | \lnot x \in A \lor \lnot x \in B\right\} \\
    & = \left\{x | x \not\in A \lor x \not\in B\right\}      \\
    & = \left\{x | x \in \bar{A} \lor x \in \bar{B}\right\} \\
    & = \left\{x | x \in \bar{A} \cup \bar{B}\right\}  \\
    & = \bar{A} \cup \bar{B}
    \end{multiequality}
}

\parag{Example: membership table}{
    Let's use a membership table to prove that:
    \[\bar{A \cap B} = \bar{A} \cup \bar{B}\]

    We can draw the following table:
    \begin{center}
    \begin{tabular}{c|c|c|c|c|c|c}
        $A$ & $B$ & $\bar{A}$ & $\bar{B}$ & $\bar{A} \cup \bar{B}$ & $A \cap B$ & $\bar{A \cap B}$ \\
        \hline
        1 & 1 & 0 & 0 & 0 & 1 & 0 \\
        1 & 0 & 0 & 1 & 1 & 0 & 1 \\
        0 & 1 & 1 & 0 & 1 & 0 & 1 \\
        0 & 0 & 1 & 1 & 1 & 0 & 1 \\
    \end{tabular}
    \end{center}


    The fifth and the seventh columns are indeed equals.
}

\parag{Identities}{
    We get the exact same identities with sets as the ones we had with propositional logic:
    \begin{center}
    \begin{tabular}{|c|c|}
        \hline
        \textbf{Identity} & \textbf{Name} \\
        \hline
        \hline
        $A \cap U = A$ & \multirow{2}{*}{Identity laws} \\
        $A \cup \o = A$ & \\
        \hline
        $A \cup U = U$ & \multirow{2}{*}{Domination laws} \\
        $A \cap \o = \o$ & \\
        \hline
        $A \cup A = A$ & \multirow{2}{*}{Idempotent laws} \\
        $A \cap A = A$ & \\
        \hline
        $\bar{\bar{A}} = A$ & Double negation law \\
        \hline
        $\bar{A \cap B} = \bar{A} \cup \bar{B}$ & \multirow{2}{*}{De Morgan's Laws} \\
        $\bar{A \cup B} = \bar{A} \cap \bar{B}$ & \\
        \hline
        \hline
        $A \cup \left(A \cap B\right) = A$ & \multirow{2}{*}{Absorption laws} \\
        $A \cap \left(A \cup B\right) = A$ & \\
        \hline
        $A \cup \bar{A} = U$ & \multirow{2}{*}{Complement laws} \\
        $A \cap \bar{A} = \o$ & \\
        \hline
        \hline
        $A \cup B = B \cup A$ & \multirow{2}{*}{Commutative laws} \\
        $A \cap B = B \cap A$ & \\
        \hline
        $\left(A \cup B\right) \cup C = A \cup \left(B \cup C\right)$ & \multirow{2}{*}{Associative laws} \\
        $\left(A \cap B\right) \cap C = A \cap \left(B \cap C\right)$ & \\
        \hline
        $A \cup \left(B \cap C\right) = \left(A \cup B\right) \cap \left(A \cup C\right)$ & \multirow{2}{*}{Distributive laws} \\
        $A \cap \left(B \cup C\right) = \left(A \cap B\right) \cup \left(A \cap C\right)$ & \\
        \hline
    \end{tabular}
    \end{center}
}

\parag{Generalised unions and intersections}{
    Since unions and intersections are associative, we can introduce the following notations. Let $A_1, \ldots, A_n$ be sets. Then:
    \[\bigcup_{i = 1}^{n} A_i = A_1 \cup \ldots \cup A_n\]
    \[\bigcap_{i = 1}^{n} A_i = A_1 \cap \ldots \cap A_n\]

    We have a similar notation for propositional logic.
}

\parag{Russel's Paradox}{
    Bertrand Russel made the following paradox.

    Let $S$ be the set that contains a set $x$ if the set $x$ does not belong to itself. In other words:
    \[S = \left\{x | x \not \in x\right\}\]

    If $S \in S$, then:
    \[S \in \left\{x |x \not \in x\right\} \implies S \not \in S\]

    Which is a contradiction. Else, if $S \not \in S$, then we have $S \in S$, which is also a contradiction.

    Those kind of paradoxes led to the building of stronger axioms for set theory, making the difference from ``naive set theory'', in which one could build any set.
}

\subsection{Function}
\parag{Definition}{
    Let $A$ and $B$ be non-empty sets. A \important{function} $f$ from $A$ to $B$ is an assignment of exactly one element of $B$ to each element of $A$, written $f: A \mapsto B$. We write $f\left(a\right) = b$ where $b$ is the unique element of $B$ assigned by the function to the element $a$ of $A$.

    Functions are sometimes called \important{mappings} or \important{transformations}.

    \subparag{Example}{
        \imagehere{functionDefinition.png}

        We will come back alter to this, but the first function is surjective, the second one is neither surjective nor injective, and the last one is injective.
    }
}

\parag{Terminology}{
    Given a function $f: A \mapsto B$, we say that $f$ \important{maps} $A$ to $B$, or $f$ is a \important{mapping} from $A$ top $B$. $A$ is called the \important{domain} of $f$ and $B$ is called the \important{codomain} of $f$.

    If $f\left(a\right) = b$, then $b$ is called the \important{image} of $a$ under $f$. $a$ is called the \important{preimage} of $b$.

    The \important{range} of $f$ is the set of all image of points in $A$ under $f$. We denote it by $f\left(A\right)$. More generally, we define $f\left(S\right)$, where $S \subseteq A$ as:
    \[f\left(S\right) = \left\{f\left(s\right) | s \in S\right\}\]

    Two functions are \important{equal} when they have the same domain, the same codomain, and map each element of the domain to the same element of the codomain.
}

\parag{Representing functions}{
    Functions may be specified in different ways. First, it can be an explicit statement of the assignment, such as a table of students and their grades. Second, it can be a formula, such as $f\left(x\right) = x + 1$. Third, it can be a computer program, such as a Python program that produces the number $2^n$ when given an integer $n$.
}

\parag{Definition: Injection}{
    A function $f$ is said to be \important{one-to-one} or \important{injective} if and only if:
    \[f\left(a\right) = f\left(b\right) \implies a = b\]

    Such a function is said to be an injection.

    \subparag{Example}{
        Every Sciper number can only be assigned to one student. If sciper(Alberts) = sciper(Karen), then we have a problem.
    }

    \subparag{Contrapositive}{
        Proving that a function may be easier using the contrapositive:
        \[a \neq b \implies f\left(a\right) \neq f\left(b\right)\]

    }
}

\parag{Definition: Surjection}{
    A function $f$ from $A$ to $B$ is called \important{onto} or \important{surjective}, if and only if:
    \[\forall b \in B \exists a \in A f\left(a\right) = b\]

    Such a function is called a \important{surjection}.

    \subparag{Example}{
        Every section has at least one student. If there is a section at EPFL which has no student, there is probably a problem.
    }
}


\parag{Definition: Bijection}{
    A function $f$ from $A$ to $B$ is a \important{one-to-one correspondance} or \important{bijection}, if it is both one-to-one and onto (surjective and injective).

    \subparag{Example}{
        Sciper numbers are bijection. Indeed, as explained before they are injective. Moreover, they are also surjective, since it would make no sense that there exists a sciper number which is assigned to no student.
    }
}

\parag{Illustration}{
    \imagehere{functionIllustration.png}

    Note that the last one is not a function since $a$ has two images.
}

\end{document}
