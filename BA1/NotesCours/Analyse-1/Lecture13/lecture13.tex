\documentclass[a4paper]{article}

% Expanded on 2021-11-03 at 10:13:00.

\usepackage{../../style}

\title{Analyse}
\author{Joachim Favre}
\date{Mercredi 03 novembre 2021}

\begin{document}
\maketitle

\lecture{13}{2021-11-03}{Définition epsilon-delta des limites}{
    \begin{itemize}[left=0pt]
        \item Définition de la composition de fonctions.
        \item Définition du voisinage et des limites de fonctions réelles.
        \item Explication et démonstration de la caractérisation de la limite d'une fonction à partir des suites.
        \item Démonstration de propriété des limites de fonctions (opérations algébriques, limites de polynômes et théorème des deux gendarmes pour les fonctions).
    \end{itemize}

}

\parag{Composition de fonctions}{
    Soient $f: E \mapsto F$ et $g : G \mapsto H$, où $E, F, G, H \subset \mathbb{R}$.

    Si $f\left(E\right) \subset G$, alors on peut définir la fonction composée:
    \[g \circ f\left(x\right) = g\left(f\left(x\right)\right) : E \mapsto H\]

    Si $f\left(G\right) \subset E$, alors on peut définir la fonction composée, avec l'autre ordre de composition:
    \[f \circ g\left(x\right) = f\left(g\left(x\right)\right) : G \mapsto F\]

    L'ordre de composition est important.
}

\parag{Exemple}{
    Soient les fonctions suivantes:
    \[\begin{split}
        f: \mathbb{R} &\longmapsto \left[-1, 1\right]  \\
        x &\longmapsto \sin x
    \end{split}
    \mathspace
    \mathspace
    \begin{split}
        g: \mathbb{R} &\longmapsto \left[-1, +\infty\right[  \\
            x &\longmapsto \sqrt{x^2 + 1}
        \end{split}
    \]

    Puisque leur domaine de définition est $\mathbb{R}$, on n'a rien besoin de vérifier. On a donc:
    \[f \circ g\left(x\right) = f\left(\sqrt{x^2 + 1}\right) = \sin\left(\sqrt{x^2 + 1}\right) : \mathbb{R} \mapsto \left[-1, 1\right]\]

    De plus:
    \[g \circ f\left(x\right) = g\left(\sin x\right) = \sqrt{\sin^2 x + 1} : \mathbb{R} \mapsto \left[1, +\infty\right[ \]
    }

    \parag{Remarque}{
        Si $f : E\mapsto F$ est bijective, alors il existe une fonction réciproque $f^{-1} : F \mapsto E$. Les conditions sont remplies pour avoir une fonction composée des deux côtés. Pour rappel, la fonction réciproque a comme définition:
        \[f^{-1}\left(y\right) = x \iff f\left(x\right) = y\]

        On a donc:
        \[f^{-1} \circ f\left(x\right) = f^{-1}\left(f\left(x\right)\right) = f^{-1}\left(y\right) = x \implies f^{-1} \circ f\left(x\right) = x \mathspace \forall x \in E\]

        De plus:
        \[f \circ f^{-1}\left(y\right) = f\left(f^{-1}\left(y\right)\right) = f\left(x\right) = y \implies f \circ f^{-1}\left(y\right) = y \mathspace \forall y \in F\]
    }

    \subsection{Limite d'une fonction}
    \parag{Définition du voisinage}{
        Une fonction $f : E \mapsto F$ est définie au \important{voisinage} de $x_0 \in \mathbb{R}$ s'il existe $\delta > 0$ tel que
        \[\left\{x \in \mathbb{R} : 0 < \left|x - x_0\right| < \delta\right\} \subset E\]

        \subparag{Remarque}{
            En d'autres mots, elle est définie partout autour de $x_0$ (mais peut ou peut ne pas être définie à $x_0$).
        }
    }

    \parag{Exemple}{
        La fonction suivante est définie au voisinage de $x_0 = 0$, mais pas en $x_0$:
        \[f\left(x\right) = \frac{\sin x}{x}\]

    }

    \parag{Définition de la limite}{
        Une fonction $f: E \mapsto F$ définie au voisinage de $x_0$ admet pour \important{limite} le nombre réel $\ell$ lorsque $x$ tend vers $x_0$, si pour tout $\epsilon > 0$ il existe $\delta > 0$ tel que pour tout $x \in E$ tels que $0 < \left|x - x_0\right| \leq \delta$, on a :
        \[\left|f\left(x\right) - \ell\right| \leq \epsilon\]

        On dit que $x$ et $x_0$ sont $\delta$-proches, et que $f\left(x\right)$ et $\ell$ sont $\epsilon$-proches.

        \subparag{Notation}{
            On écrit:
            \[\lim_{x \to x_0} f\left(x\right) = \ell\]
        }

        \subparag{Intuition}{
            Ce que cette définition veut dire c'est que pour n'importe quel écart vertical à la valeur de la limite, $\epsilon$, alors il existe un ensemble centré sur $x_0$ tel que toutes les images de ces $x$ soit dans l'écart vertical.

            \imagehere[0.5]{DefinitionLimite.png}

            Il y a une très bonne vidéo de 3Blue1Brown qui parle de l'intuition derrière cette définition :
            \begin{center}
                \url{https://www.youtube.com/watch?v=kfF40MiS7zA}.
            \end{center}
        }
    }

    \parag{Exemple}{
        Soit la fonction $f\left(x\right) = 3x - 1$, avec $x_0 = 1$. On veut démontrer que
        \[\lim_{x \to 1} \left(3x - 1\right) = 2\]

        Soit $\epsilon > 0$. Il nous faut trouver $\delta > 0$ tel que:
        \[0 < \left|x - 1\right| \leq \delta \implies \left|f\left(x\right) - 2\right| \leq \epsilon\]

        Étudions l'inégalité suivante:
        \[\left|f\left(x\right) - 2\right| = \left|3x - 1 - 2\right| = \left|3x - 3\right| = 3\left|x -  1\right| \leq 3\delta\]

        Or, on veut que $\left|f\left(x\right) - 2\right| \leq \epsilon$. On peut donc prendre $\delta = \frac{\epsilon}{3}$:
        \[\left|x - 1\right| \leq \delta = \frac{\epsilon}{3} \implies \left|f\left(x\right) - 2\right| = 3\left|x - 1\right| \leq 3\delta = 3 \frac{\epsilon}{3} = \epsilon\]

        Donc, en d'autres mots, $\forall \epsilon > 0$ on a trouvé $\delta = \frac{\epsilon}{3}$ tel que la définition de la limite $\lim_{x \to 1} f\left(x\right) = 2$ est satisfaite. On a donc démontré que:
        \[\lim_{x \to 1} \left(3x - 1\right) = 2\]

        \qed

    }

    \parag{Proposition (Caractérisation de la limite d'une fonction à partir des suites)}{
        Soit $f : E \mapsto F$, et $A$ l'ensemble des suites $\left(a_n\right) \subset E \setminus \left\{x_0\right\}$ telles que $\lim_{n \to \infty} a_n = x_0$. On a :

        \[\lim_{x \to x_0} f\left(x\right) = \ell \iff \forall \left(a_n\right) \in A, \lim_{n \to \infty} f\left(a_n\right) = \ell\]

        \subparag{Preuve si}{
            Soit $\epsilon > 0$. On sait que $\exists \delta$ tel que, pour tout $x$:
            \[0 < \left|x - x_0\right| \leq \delta \implies \left|f\left(x\right) - \ell\right| \leq \epsilon\]

            Soit $\left(a_n\right)$, une suite dont la limite est $x_0$, et telle que $a_n \neq x_0$ pour tout $n$. Puisqu'on sait que sa limite est $x_0$, on sait que $\exists n_0 \in \mathbb{N}$ tel que $\forall n \geq n_0$:
            \[\left|a_n - x_0\right| \leq \delta\]

            En prenant $x = a_n$, cela implique donc par la première inégalité ci-dessus que
            \[\forall n \geq n_0,\ \left|f\left(a_n\right) - \ell\right| \leq \epsilon \over{\implies}{déf} \lim_{n \to \infty} f\left(a_n\right) = \ell\]
        }

        \subparag{Seulement si}{
            Nous allons faire une preuve par la contraposée. Nous supposons donc que $\lim_{x \to x_0} f\left(x\right) \neq \ell$, et nous voulons montrer qu'il existe une suite $\left(a_n\right) \in A$ telle que $\lim_{n \to \infty} f\left(a_n\right) \neq \ell$.

            La fonction n'admet pas pour limite $\ell$ lorsque $x$ tend vers $x_0$, donc $\exists \epsilon > 0$ tel que $\forall \delta > 0$ il existe un $x$ pour lequel:
            \[0 < \left|x - x_0\right| \leq \delta \mathspace \text{et} \mathspace \left|f\left(x\right) - \ell\right| > \epsilon\]

            Alors, on va construire une suite $\left(a_n\right) \subset E \setminus \left\{x_0\right\}$ telle que $\lim_{n \to \infty} a_n = x_0$, mais $\lim_{n \to \infty} f\left(a_n\right) \neq \ell$.

            Prenons le $\epsilon > 0$ dont on connait l'existence. Puisque l'inégalité ci-dessus est vraie pour tout $\delta > 0$, on peut construire la suite suivante :
            \[\delta = 1 \implies \exists a_1 \in E \telque 0 < \left|a_1 - x_0\right| \leq 1 \text{ et } \left|f\left(a_1\right) - \ell\right| > \epsilon\]
            \[\delta = \frac{1}{2} \implies \exists a_2 \in E \telque 0 < \left|a_2 - x_0\right| \leq \frac{1}{2} \text{ et } \left|f\left(a_2\right) - \ell\right| > \epsilon\]

            On continue comme ça jusqu'à l'infini. Ainsi, le $n$-ème terme de notre suite est donné par:
            \[\delta = \frac{1}{2^{n-1}} \implies \exists a_n \in E \telque 0 < \left|a_n - x_0\right| \leq \frac{1}{2^{n-1}}\]

            Ainsi on voit que, pour n'importe quel $\epsilon' > 0$, on peut trouver un $n_0$ tel que pour tout $n \geq n_0$, alors $\left|a_n - x_0\right| \leq \epsilon'$. Donc, par la définition de la limite:
            \[\lim_{n \to \infty} a_n = x_0\]

            Or, pour tout $n \in \mathbb{N}$, on avait trouvé que $\left|f\left(a_n\right) - \ell\right| > \epsilon$. Donc, nécessairement:
            \[\lim_{n \to \infty} f\left(a_n\right) \neq \ell\]

        }
    }

    \parag{Remarque}{
        ``Toute suite'' dans la définition est très important.

        En effet, soit la fonction suivante:
        \begin{functionbypart}{f\left(x\right)}
            1, \mathspace \text{si } x = \frac{1}{n}, n \in \mathbb{N}^* \\
            0, \mathspace \text{sinon}
        \end{functionbypart}

        On remarque que $\lim_{x \to 0} f\left(x\right)$ n'existe pas. Cependant, la suite $\left(a_n = \frac{1}{n}, n \in \mathbb{N}^*\right)$ est telle que:
        \[\lim_{n \to \infty} f\left(a_n\right) = \lim_{n \to \infty} f\left(\frac{1}{n}\right) = 1 \mathspace \text{et} \mathspace \lim_{n \to \infty} a_n = \lim_{n \to \infty} \frac{1}{n} = 0\]
    }

    \parag{Corollaire}{
        Soit $f  : E \mapsto F$ définie au voisinage de $x_0$. Supposons que pour toute suite $\left(a_n\right) \subset E \setminus \left\{x_0\right\}$ telle que $\lim_{n \to \infty} a_n = x_0$, la suite $\left(f\left(a_n\right)\right)$ converge.

        Alors, $\lim_{x \to x_0} f\left(x\right)$ existe.

        \subparag{Preuve}{
            Supposons par l'absurde que $\lim_{x \to x_0} f\left(x\right)$ n'existe pas. En d'autres mots, on sait par la caractérisation des suites qu'il existe deux suites $\left(a_n\right), \left(b_n\right) \subset E \setminus \left\{x_0\right\}$ telles que $\lim_{n \to \infty} a_n = \lim_{n \to \infty} b_n = x_0$, mais:
            \[\lim_{n \to \infty} f\left(a_n\right) \neq \lim_{n \to \infty} f\left(b_n\right)\]

            Soit la suite suivante:
            \begin{functionbypart}{c_n}
                a_{n/2}, \mathspace n \text{ pair} \\
                b_{\left(n-1\right)/2}, \mathspace n \text{ impair}
            \end{functionbypart}

            On remarque que $\lim_{n \to \infty} c_n = x_0$, puisque toutes ses sous-suites convergent vers $x_0$. De plus, par notre hypothèse:
            \[\lim_{n \to \infty} f\left(c_{2n}\right) = \lim_{n \to \infty} f\left(a_{n}\right) \neq \lim_{n \to \infty} f\left(b_n\right) = \lim_{n \to \infty} f\left(c_{2n + 1}\right)\]

            Cependant, cela veut donc dire que:
            \[\lim_{n \to \infty} f\left(c_{2n}\right) \neq \lim_{n \to \infty} f\left(c_{2n + 1}\right)\]

            Or, par la contraposée du théorème de la convergence des sous-suites, cela veut dire que $f\left(c_n\right)$ ne converge pas, ce qui est une contradiction puisqu'elle devrait converger par hypothèse, car $c_n$ converge vers $x_0$.

            \qed
        }
    }

    \parag{Exemple}{
        Soit la fonction $f\left(x\right) = x^{p}$ où $p \in \mathbb{N}^*$. On veut montrer que
        \[\lim_{x \to x_0} x^{p} = x_0^p \mathspace \forall x_0 \in \mathbb{R}\]

        Soit $\left(x_n\right)$ une suite arbitraire convergente vers $x_0$ où $x_n \neq x_0$ pour tout $n \in \mathbb{N}$. Soit $a_n = x_n - x_0$ pour tout $n \in \mathbb{N}$. On a donc:
        \[\lim_{n \to \infty} a_n = \lim_{n \to \infty} \left(x_n - x_0\right) = \lim_{n \to \infty} x_n - x_0 = x_0 - x_0 = 0\]

        Donc:
        \begin{multiequality}
            \lim_{n \to \infty} f\left(x_n\right) & = \lim_{n \to \infty} f\left(x_0 + a_n\right)  \\
                                                  & = \lim_{n \to \infty} \left(x_0 + a_n\right)^{p}  \\
                                                  & = \lim_{n \to \infty} \left(x_0^p + \underbrace{\binom{p}{1}x_0^{p-1}a_n}_{\to 0} + \ldots + \underbrace{\binom{p}{p}a_n^{p}}_{\to0}\right)  \\
                                                  & = x_0^p
        \end{multiequality}


        On en déduit que, par la caractérisation:
        \[\lim_{x \to x_0} x^{p} = x_0^p\]

    }

    \parag{Proposition (unicité de la limite)}{
        Si $\lim_{x \to x_0} f\left(x\right) = \ell_1$ et $\lim_{x \to x_0} f\left(x\right) = \ell_2$, alors:
        \[\ell_1 = \ell_2\]

        \subparag{Preuve}{
            Supposons que:
            \[\lim_{x \to x_0} f\left(x\right) = \ell_1 \mathspace \text{ et } \mathspace \lim_{x \to x_0} f\left(x\right) = \ell_2\]

            Donc, par la caractérisation des suites, on sait que toutes suites $\left(a_n\right) \subset E \setminus \left\{x_0\right\}$ convergentes vers $x_0$, on a:
            \[\lim_{n \to \infty} f\left(a_n\right) = \ell_1 \mathspace \text{ et } \mathspace \lim_{n \to \infty} f\left(a_n\right) = \ell_2\]

            Par l'unicité de la limite d'une suite, on a $\ell_1 = \ell_2$.

            \qed
        }
    }

    \parag{Proposition (Critère de Cauchy pour les fonctions)}{
        $\exists \lim_{x \to x_0} f\left(x\right)$ si et seulement si pour tout $\epsilon > 0$, il existe $\delta > 0$ tels que
        \[\forall x_1, x_2 \in \left\{x \in E \telque 0 < \left|x - x_0\right| \leq \delta\right\} \implies \left|f\left(x_1\right) - f\left(x_2\right)\right| \leq \epsilon\]

        \subparag{Autre formulation}{
            En d'autres mots, pour tout écart vertical arbitraire à la limite, il existe un écart horizontal à $x_0$, tel que pour toute paire de nombres $x_1$ et $x_2$ $\delta$-proche de $x_0$, alors elles leur image sont $\epsilon$-proches.
        }
    }

    \parag{Proposition (opérations algébriques sur les limites)}{
        Soient $f: E\mapsto \mathbb{R}$ et $g: E \mapsto \mathbb{R}$ telles que:
        \[\lim_{x \to x_0} f\left(x\right) = \ell_1 \in \mathbb{R} \mathspace \text{ et } \mathspace \lim_{x \to x_0} g\left(x\right) = \ell_2 \in \mathbb{R}\]

        Alors:
        \begin{itemize}
            \item $\lim\limits_{x \to x_0} \left(\alpha f\left(x\right) + \beta g\left(x\right)\right) = \alpha \ell_1 + \beta \ell_2$
            \item $\lim\limits_{x \to x_0} \left(f\left(x\right)g\left(x\right)\right) = \ell_1 \ell_2$
            \item $\lim\limits_{x \to x_0} \left(\frac{f\left(x\right)}{g\left(x\right)}\right) = \frac{\ell_1}{\ell_2}$ si $\ell_2 \neq 0$ et $g\left(x\right) \neq 0$ au voisinage de $x_0$.
        \end{itemize}

        \subparag{Preuve}{
            On peut utiliser la caractérisation à partir des suites.
        }

    }

    \parag{Limite de polynômes}{
        Pour tout polynôme et fonction rationnelle,
        \[\lim_{x \to x_0} f\left(x\right) = f\left(x_0\right) \]
        pour tout $x_0 \in \mathbb{R}$ sauf les zéros du dénominateur de $f\left(x\right)$.

        \subparag{Idée de preuve}{
            On a démontré que
            \[\lim_{x \to x_0} x^{p} = x_0^p \mathspace \forall p \in \mathbb{N}^*, \forall x_0 \in \mathbb{R}\]

            On peut utiliser ce fait avec les opération algébriques.
        }

        \subparag{Exemple}{
            \[\lim_{x \to 2} \frac{3x - 4}{x^2 + 2x + 5} = \frac{3\left(2\right) - 4}{\left(2\right)^2 + 2\left(2\right) + 5} = \frac{2}{13}\]
        }

        \subparag{Continuité}{
            Ce fait implique que ces fonctions sont continues partout sauf aux zéros de leur dénominateur.
        }
    }

    \parag{Théorème des deux gendarmes pour les fonctions}{
        Soient $f,g,h : E \mapsto F$ telles que:
        \begin{enumerate}
            \item $\lim\limits_{x \to x_0} f\left(x\right) = \lim\limits_{x \to x_0} g\left(x\right) = \ell$
            \item $\exists \alpha > 0$ tel que $\forall x \in \left\{x \in E \telque 0 < \left|x - x_0\right| \leq \alpha\right\}$ on a:
                \[f\left(x\right) \leq h\left(x\right) \leq g\left(x\right)\]

                En d'autres mots, il existe un $\alpha$ tels que pour tout $x$ $\alpha$-proche de $x_0$, la relation d'ordre ci-dessus est tenue.
        \end{enumerate}

        Alors, si ces deux propriétés sont tenues,
        \[\lim_{x \to x_0} h\left(x\right) = \ell\]

        \imagehere[0.5]{Theoreme2GendarmesFonctions.png}

        \subparag{Preuve}{
            Soit la suite arbitraire suivante:
            \[\left(a_n\right) \subset E \setminus \left\{x_0\right\}, \mathspace \lim_{n \to \infty} a_n = x_0\]

            Par la définition de la limite des suites, en prenant $\epsilon = \alpha$, on sait qu'il existe $m \in \mathbb{N}$ tel que:
            \[\left|a_n - x_0\right| \leq \alpha, \mathspace \forall n \geq m\]

            Ainsi, puisque $a_n$ est $\alpha$-proche de $x_0$ pour tout $n \geq m$, on peut en déduire par notre deuxième hypothèse:
            \[f\left(a_n\right) \leq h\left(a_n\right) \leq g\left(a_n\right) \mathspace \forall n \geq m\]

            Puisque $\lim_{x \to x_0} f\left(x\right) = \lim_{x \to x_0} g\left(x\right) = \ell$, on a par la caractérisation à partir des suites que:
            \[\lim_{n \to \infty} f\left(a_n\right) = \lim_{n \to \infty} g\left(a_n\right) = \ell\]

            On peut en déduire par les deux gendarmes pour les suites que:
            \[\lim_{n \to \infty} h\left(a_n\right) = \ell\]

            Or, puisqu'on a pris une suite $\left(a_n\right) \subset E \setminus \left\{x_0\right\}$ telle que $\lim_{n \to \infty} a_n = x_0$ quelconque, on sait que c'est vrai pour tout suite $\left(a_n\right)$ suivant ces hypothèses. Ainsi, par la caractérisation à partir des suites:
            \[\lim_{x \to x_0} h\left(x\right) = \ell\]

            \qed
        }

    }


    \end{document}
