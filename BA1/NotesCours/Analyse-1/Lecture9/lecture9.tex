\documentclass[a4paper]{article}

% Expanded on 2021-10-20 at 10:07:51.

\usepackage{../../style}

\title{Analyse 1}
\author{Joachim Favre}
\date{Mercredi 20 octobre 2021}

\begin{document}
\maketitle

\lecture{9}{2021-10-20}{Constante d'Euler, récurrence et sous-suites}{
    \begin{itemize}[left=0pt]
        \item Définition de la constante d'Euler, $e$.
        \item Définition des suites définies par récurrence.
        \item Dérivation des outils pour analyser ces dernières (récurrence linéaire, relation de récurrence croissante, etc.)
        \item Définition des sous-suites.
        \item Explication du théorème de la convergence des sous-suites, et du théorème de Bolzano-Weierstrass.
    \end{itemize}

}

\subsection{Le nombre $e$}
\parag{Proposition}{
    Soient $\left(x_n\right)$ et $\left(y_n\right)$ deux suites définies telles que:
    \[x_0 = y_0 = 1 \mathspace x_n = \left(1 + \frac{1}{n}\right)^n \mathspace y_n = 1 + \frac{1}{1!} + \frac{1}{2!} + \ldots + \frac{1}{n!} \mathspace \forall n \geq 1\]

    Alors:
    \begin{enumerate}
        \item $x_n \leq y_n$, $\forall n \in \mathbb{N}$
        \item $y_n \leq 3$, $\forall n \in \mathbb{N}$
        \item $\left(y_n\right)\uparrow$, $\forall n \in \mathbb{N}$
        \item $\left(x_n\right)\uparrow$, $\forall n \in \mathbb{N}$
    \end{enumerate}

    De (2) et (3) on peut en déduire que $\left(y_n\right)$ converge et que sa limite est plus petite ou égale à 3. Par là, et (1) et (4) on en déduit que $\left(x_n\right)$ converge, vers une valeur plus petite ou égale à la limite de $\left(y_n\right)$, donc plus petite ou égale à 3.

    \subparag{Preuve du point (1)}{
        On sait que $x_0 \leq y_0$. De plus, par le binôme de Newton:
        \[x_n = \left(1 + \frac{1}{n}\right)^{n} = 1 + \binom{n}{1} \frac{1}{n} + \ldots + \binom{n}{k} \left(\frac{1}{n}\right)^{k} + \ldots + \binom{n}{n} \left(\frac{1}{n}\right)^{n}\]

        On veut démontrer que chaque terme de $\left(x_n\right)$ est plus petit que chaque terme de $\left(y_n\right)$ (cela nous permettrait de conclure ce qu'on veut conclure, que $x_n \leq y_n$):
        \[\binom{n}{k} \left(\frac{1}{n}\right)^{k} = \frac{n!}{k! \left(n-k\right)!} \frac{1}{n^{k}} = \frac{1}{k!} \underbrace{\frac{n}{n}}_{\leq1} \underbrace{\frac{n-1}{n}}_{<1} \ldots \underbrace{\frac{n - k + 1}{n}}_{< 1} \leq \frac{1}{k!}\]

        Donc,
        \[x_n = 1 + \binom{n}{1} \frac{1}{n} + \ldots + \binom{n}{n} \left(\frac{1}{n}\right)^{n} \leq 1 + \frac{1}{1!} + \ldots + \frac{1}{n!} = y_n\]
        pour tout $n \geq 1$.
    }

    \subparag{Preuve du point (2)}{
        On veut démontrer par récurrence que:
        \[\frac{1}{k!} \leq \frac{1}{2^{k - 1}} \mathspace \forall k \geq 2\]

        \important{Initialisation:} On sait que
        \[\frac{1}{2!} = \frac{1}{2} \leq \frac{1}{2^{1}} = \frac{1}{2}\]

        \important{Hérédité:} Partons du côté gauche:
        \[\frac{1}{k!} = \underbrace{\frac{1}{\left(k-1\right)!}}_{\text{supposition}} \cdot \underbrace{\frac{1}{k}}_{\leq \frac{1}{2}} \leq \frac{1}{2^{k-2}} \cdot \frac{1}{2} = \frac{1}{2^{k-1}}\]

        \important{Suite de la preuve:} On sait maintenant que
        \[y_n = 1 + \frac{1}{1!} + \ldots + \frac{1}{n!} \leq 1 + \frac{1}{1} + \ldots + \frac{1}{2^{n-1}}\]

        De plus, puisque
        \[1 + x + \ldots + x^{n-1} = \frac{1 - x^{n}}{1 - x} \mathspace \forall x \neq 1\]
        on peut prendre $x = \frac{1}{2}$:
        \[y_n \leq 1 + \frac{1}{1} + \frac{1}{2} + \ldots + \frac{1}{2^{n-1}} = 1 + \frac{1 - \left(\frac{1}{2}\right)^{n}}{1 - \frac{1}{2}} < 1 + \frac{1}{1 - \frac{1}{2}} = 3 \mathspace \forall n \in \mathbb{N}\]
    }

    \subparag{Preuve du point (3)}{
        On remarque que:
        \[y_{n+1} = 1 + \frac{1}{1!} + \frac{1}{2!} + \ldots + \frac{1}{n!} + \frac{1}{\left(n+1\right)!} = y_n + \frac{1}{\left(n+1\right)!} > x_n\]

        Donc $\left(y_n\right)\uparrow$.
    }

    \subparag{Preuve du point (4)}{
        Par l'inégalité AM--GM (moyenne arithmétique--moyenne géométrique), on sait que
        \[\sqrt[n+1]{a_0 \cdot a_1 \cdots a_n} \leq \frac{a_0 + \ldots + a_n}{n + 1}\]

        En prenant $a_0 = 1$ et $a_1 = a_2 = \ldots = a_n = 1 + \frac{1}{n}$, on a :
        \[\sqrt[n+1]{\left(1 + \frac{1}{n}\right)^{n}} \leq \frac{1 + n\left(1 + \frac{1}{n}\right)}{n+1} = \frac{1 + n + 1}{n+1} = 1 + \frac{1}{n+1}\]

        Donc, en mettant à la puissance $n+1$ des deux côtés:
        \[\left(1 + \frac{1}{n}\right)^{n} \leq \left(1 + \frac{1}{n+1}\right)^{n+1}\]

        \qed
    }
}

\parag{Définition de $e$}{
    On définit:
    \[\lim_{n \to \infty} \left(1 + \frac{1}{n}\right)^{n} \equiv e\]

    \subparag{Remarque}{
        On peut démontrer (voir le livre) que
        \[\lim_{n \to \infty} y_n = e\]
    }

    \subparag{Estimation}{
        On a:
        \[e \approx 2.718281828459045\ldots\]
    }
}


\subsection{Suites définies par récurrence}
\parag{Introduction}{
    Soit $x_0 = a \in \mathbb{R}$, et $x_{n + 1} = g\left(x_n\right)$ où $g: \mathbb{R} \mapsto \mathbb{R}$ est une fonction.

    Est-ce que la suite converge ? Si oui, quelle est sa limite ?
}

\parag{Exemple 1}{
    Soit la suite $\left(x_n\right)$ définie par récurrence:
    \[x_0 = 1; \mathspace x_{n+1} = 5 + \frac{6}{x_n}\]

    On peut trouver les premiers termes de la suite
    \[x_0 = 1; \mathspace x_1 = 5 + \frac{6}{1} = 11; \mathspace x_2 = 5 + \frac{6}{11}; \mathspace \ldots\]

    \imagehere[0.5]{SuitesRecursivesExemple1.png}

    \subparag{Possibilités de limites}{
        Supposons que
        \[\lim_{n \to \infty} x_n = \ell \neq 0\]

        Alors, $\frac{1}{x_n}$ converge vers $\frac{1}{\ell}$. On obtient l'équation suivante pour $\ell$:
        \[\ell = 5 + \frac{6}{\ell} \implies \ell^2 - 5\ell - 6 = 0 \implies \ell = \frac{5 \pm 7}{2} \implies \ell = 6, \ell = -1\]
    }


    \subparag{Borne inférieure}{
        On veut démontrer que $x_n > 5$ pour tout $n \geq 1$ par récurrence:
        \[x_1 = 11 > 5\]
        qui est vrai. De plus:
        \[x_{n + 1} = 5 + \frac{6}{\underbrace{x_n}_{>0}} > 5\]

        On peut donc en déduire que si la suite converge, la limite est 6 (ça ne peut pas converger vers $-1$).
    }

    \subparag{Démonstration convergence}{
        On veut démontrer que ça converge bien vers notre $\ell$:
        \[\left|x_{n+1} - \ell\right| = \left|5 + \frac{6}{x_n} - \left(5 + \frac{6}{\ell}\right)\right| = \left|\frac{6}{x_n} - \frac{6}{\ell}\right| = \frac{6\left|l - x_n\right|}{\left|x_n\right|\left|\ell\right|} < \frac{6}{25}\left|x_n - \ell\right|\]

        De là, on en déduit que
        \[\left|x_{n+1} - \ell\right| < \frac{6}{25} \left|x_n - \ell\right| < \left(\frac{6}{25}\right)^2 \left|x_{n-1} - \ell\right| < \ldots < \left(\frac{6}{25}\right)^{n} \left|x_1 - \ell\right|\]

        Donc, on a que:
        \[0 \leq \left|x_{n+1} - \ell\right| < \left(\frac{6}{25}\right)^{n} \left|x_1 - \ell\right|\]
        Par les deux gendarmes, on a que
        \[\lim_{n \to \infty} \left|x_{n+1} - \ell\right| = 0 \implies \lim_{n \to \infty} \left(x_{n+1} - \ell\right) = 0 \implies \lim_{n \to \infty} x_n = \ell = 6\]

    }

}

\parag{Proposition (récurrence linéaire)}{
    Soit $a_0 \in \mathbb{R}$, et avec $a_{n+1} = q a_n + b$, où $q, b \in \mathbb{R}$. Alors:
    \begin{enumerate}
        \item Si $\left|q\right| < 1$, alors $\left(a_n\right)$ converge vers vers
            \[\lim_{n \to \infty} a_n = \frac{b}{1-q}\]
        \item Si $\left|q\right| \geq 1$, alors $\left(a_n\right)$ diverge, sauf si $\left(a_n\right)$ est une suite constante.
    \end{enumerate}

    \imagehere{PreuveRecurrenceLineaire.png}

    \subparag{Preuve}{
        Supposons que $\left(a_n\right)$ converge. Alors, l'équation pour la limite est
        \[\ell = q \ell + b \implies \ell = \frac{b}{1-q}\]

        \important{Quand $\left|q\right| < 1$:} On a pour tout $n \geq 1$:
        \[0 \leq \left|a_{n+1} - \ell\right| = \left|q a_n + b - \left(q\ell + b\right)\right| = \left|q\right|\left|a_n - \ell\right| = \ldots = \left|q\right|^{n+1}\left|a_0 - \ell\right|\]

        Or, $\left|q\right|^{n+1}\left|a_0 - \ell\right|$ converge vers 0. Donc, par les deux gendarmes, on a que
        \[\lim_{n \to \infty} \left|a_{n+1} - \ell\right| = 0 \implies \lim_{n \to \infty} a_n = \frac{b}{1-q}\]

        \vspace{1em}
        \important{Quand $\left|q\right| \geq 1$:} On a démontré ci-dessus que
        \[\left|a_{n+1} - \ell\right| = \left|q\right|^{n+1} \left|a_0 - \ell\right|\]

        Or, puisque $\left|q\right| \geq 1$, on sait que la limite diverge ($a_0 - \ell \neq 0$).

        \vspace{1em}
        \important{Quand $q = 1$ ou $q = -1$:} La preuve est laissée en exercice au lecteur. Indice : faire le cas où $q = 1$, et le cas où $q = -1$.
    }
}

\parag{Proposition}{
    Soit $x_0 \in \mathbb{R}$, $x_{n+1} = g\left(x_n\right)$ et $g : E \mapsto E \subset \mathbb{R}$.

    \begin{itemize}
        \item Si $g$ est bornée, c'est-à-dire que $\exists m, M \in \mathbb{R}$ tels que
            \[m \leq g\left(x\right) \leq M \mathspace \forall x \in E\]
            alors la suite est bornée.

        \item Si $g$ est croissante, c'est-à-dire que $\forall x_1, x_2 \in E$:
            \[x_1 \leq x_2 \implies g\left(x_1\right) < g\left(x_2\right)\]
            alors la suite est monotone.
    \end{itemize}

    \subparag{Remarque}{
        Si $g$ est décroissante, alors $\left(x_n\right)$ n'est pas monotone, on en est sûr (mais la suite peut être convergente). Par exemple, le premier exemple qu'on a vu.
    }

    \subparag{Preuve}{
        Faite en exercice dans la série 6.
    }

}

\parag{Exemple}{
    On prend une suite de très proche de celle de notre premier exemple:
    \[x_0 = 4; \mathspace x_{n+1} = 5 - \frac{6}{x_n}\]

    On remarque que $g$ est croissante. Donc, $\left(x_n\right)$ est monotone. On peut démontrer par récurrence que, en effet, elle est décroissante:
    \[x_0 = 4; \mathspace x_1 = 5 - \frac{6}{4} < x_0\]

    De plus, l'hérédité est aussi vraie:
    \[x_{n+1} < x_n \implies \underbrace{ g\left(x_{n+1}\right)}_{x_{n+2}} < \underbrace{g\left(x_n\right)}_{x_{n+1}} \implies x_{n+2} < x_{n+1}\]

    Donc, $\left(x_n\right)\downarrow$ par récurrence.

    On remarque que, si $x \geq 3$, alors:
    \[g\left(x\right) = 5 - \frac{6}{x} \geq 5 - 2 = 3\]

    Or, puisque $x_0 = 4 > 3$, alors on sait que $x_n \geq 3$ pour tout $n$. Puisque $\left(x_n\right)$ est minorée par $3$ et que $\left(x_n\right)\downarrow$, alors
    \[\exists \lim_{n \to \infty} x_n \geq 3\]

    On peut utiliser l'équation pour la limite:
    \[\ell = 5 - \frac{6}{\ell} \implies \ell^2 - 5\ell + 6 = 0 \implies \ell = 3; 2\]

    Puisque $\left(x_n\right) \geq 3$ pour tot $n \in \mathbb{N}$, alors on sait que
    \[\lim_{n \to \infty} x_n = 3\]

}

\parag{Astuces pour étudier les suites définies par récurrence}{
    \begin{enumerate}[left=0pt]
        \item Trouver les candidats pour la limite, en supposant que cette dernière existe. Si cette équation n'admet pas de solution, alors la suite diverge.
        \item Étudier la convergence.
            \begin{enumerate}
                \item Récurrence linéaire, c'est à dire que $x_{n + 1} = q x_n + b$. Alors:
                    \begin{itemize}
                        \item Si $\left|q\right| < 1$, alors
                            \[\lim_{n \to \infty} x_n = \frac{b}{1 - q}\]
                        \item Si $\left|q\right| > 1$, alors $\left(x_n\right)$ diverge (sauf si $x_n$ est une constante, c'est à dire que $x_0 = \frac{b}{1 -q}$).
                        \item Si $\left|q\right| = 1$, alors $\left(x_n\right)$ diverge (sauf si $\left(x_n\right)$ est constante).
                    \end{itemize}

                \item Si $x_{n+1} = g\left(x_n\right)$ avec $g$ croissante. Alors, on sait que la suite est monotone.
                    \begin{itemize}
                        \item Si $x_0 < x_1$, alors la suite est croissante, et on veut montrer qu'elle est majorée, auquel cas elle converge.
                        \item Si $x_0 > x_1$, alors la suite est décroissante, et on veut montrer qu'elle est minorée, auquel cas elle converge.
                    \end{itemize}

                \item Si $x_{n+1} = g\left(x_n\right)$, mais que $g$ n'est ni linéaire ni croissante. Alors, on peut faire un graphique pour se donner une idée. On peut essayer d'utiliser la proposition suivante.

                \item Démontrer que $\left(x_n\right)$ est une suite de Cauchy (qu'on définira plus tard).
            \end{enumerate}
    \end{enumerate}
}

\parag{Proposition}{
    Si $\left(x_n\right)$ et $\left(a_n\right)$ deux suites, où
    \[0 < a_n < 1 \mathspace \forall n \in \mathbb{N}\]
    et que $\exists \ell \in \mathbb{R}$ tel que
    \[\left(x_{n+1} - \ell\right) = a_n \left(x_n - \ell\right)\]
    alors $\left(x_n\right)$ converge.

    De la même manière, si $\left|x_{n+1} - \ell\right| \leq b_n \left|x_n - \ell\right|$ et $0 < b_n < \rho < 1$, alors $x_n$ converge et $\lim_{n \to \infty} x_n = \ell$

    \subparag{Note}{
        Le $\rho$ dans l'inégalité est important pour dire que $\left(b_n\right)$ ne converge pas vers 1. Une notation équivalente serait de dire que $0 < b_n < 1$ et $\left(b_n\right)$ n'admet pas de sous-suite qui tende vers $1$. Sinon, un contre-exemple serait $b_n = 1 - \frac{1}{n}$.

        C'est d'ailleurs la différence majeure entre la première affirmation et la deuxième (en plus des valeurs absolues). La première affirmation est inclue dans la deuxième : on ne sait pas vers quoi converge la première affirmation, cependant si on avait $0 < a_n < \rho < 1$, alors on saurait que sa limite est $\ell$.
    }
}

\subsection{Sous-suites}
\parag{Définition des sous-suites}{
    Une \important{sous-suite} d'une suite $\left(a_n\right)$ est une suite $k \mapsto a_{n_k}$ où $k \mapsto n_k$ est une suite \emph{strictement} croissante de nombres naturels.
}

\parag{Exemple}{
    Prenons $a_n = \left(-1\right)^{n}$ pour tout $n \in \mathbb{N}$. Alors
    \[a_{2n} = \left(-1\right)^{2k} = 1 \mathspace \text{et} \mathspace a_{2k+1} = \left(-1\right)^{2k+1} = -1\]

    On remarque que
    \[\lim_{n \to \infty} a_{2n} = 1 \mathspace \text{et} \lim_{n \to \infty} a_{2n+1} = -1\]
    alors que $\left(a_n\right)$ est divergente.
}

\parag{Proposition (convergence d'une sous-suite)}{
    Si $\left(a_n\right)$ converge vers une limite $\ell$, alors toute sous suite $\left(a_{n_{k}}\right)$ converge aussi vers $\ell$.

    \subparag{Démonstration}{
        Soit $\epsilon > 0$. On sait que $\exists n_0 \in \mathbb{N}$ tel que $\forall n \geq n_0$, alors:
        \[\left|a_{n} - \ell\right| \leq \epsilon\]


        Donc, $\forall k \geq n_0$, on a $n_k \geq k \geq n_0$, ce qui veut dire que
        \[\left|a_{n_k} - \ell\right| \leq \epsilon \implies \lim_{k \to \infty} a_{n_k} = \ell\]
    }

    \subparag{Note personnelle: contraposée}{
        La contraposée de ce théorème est que s'il existe deux sous-suites de $\left(a_n\right)$ qui convergent vers des valeurs différentes, alors $\left(a_n\right)$ est divergente.
    }

}

\parag{Théorème de Bolzano-Weierstrass}{
    Dans toute suite bornée, il existe une sous-suite convergente. En d'autres mots, si on a $\left(a_n\right)$ tel que
    \[\exists m, M \telque m \leq a_n \leq M \implies \exists \left(a_{n_k}\right) \subset \left(a_n\right) \telque \lim_{k \to \infty} a_{n_k} = \ell \in \mathbb{R}\]


    \subparag{Idée de la preuve}{
        Soit $\left(a_n\right)$, une suite bornée. De ce fait $\exists m, M \in \mathbb{R}$ tels que
        \[m \leq a_n \leq M \mathspace \forall n \in \mathbb{N}\]

        On divise l'intervalle $\left[m,M\right]$ par 2. On retient la moitié contenant un nombre infini d'éléments de $\left(a_n\right)$ (si les deux en contiennent un nombre infini, on en choisi une). Puis, on recommence. La longueur des intervalles est donnée par $\frac{M - m}{2^{n}}$ (où $n$ est le nombre d'itération) qui tend vers 0. En choisissant un élément dans chaque intervalle, à chaque fois de telle manière à ce qu'il soit plus loin dans la suite, on obtient une sous-suite convergente.
    }

}







\end{document}
