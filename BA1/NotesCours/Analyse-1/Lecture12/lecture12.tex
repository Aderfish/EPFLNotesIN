\documentclass[a4paper]{article}

% Expanded on 2021-11-01 at 10:11:24.

\usepackage{../../style}

\title{Analyse 1}
\author{Joachim Favre}
\date{Lundi 01 novembre 2021}

\begin{document}
\maketitle

\lecture{12}{2021-11-01}{Le cours le plus simple, selon la Professeure}{
\begin{itemize}[left=0pt]
    \item Définition du concept de fonction, et de celui de graphique.
    \item Définition des différentes propriétés que les fonctions peuvent avoir (croissante, strictement croissante, décroissante, strictement décroissante, monotone, strictement monotone, paire, impaire, périodique, majorée, minorée, bornée, surjective, injective, bijective).
    \item Définition de la borne supérieure, la borne inférieure, le minimum local, maximum local, minimum global et maximum global.
    \item Explication de comment trouver la plus petite période d'une fonction périodique (si la plus petite période existe).
    \item Explication de comment rendre une fonction bijective, et comment trouver son inverse.
\end{itemize}

}

\section{Fonctions réelles}
\subsection{Définitions et propriétés de bases}
\parag{Définition des fonctions}{
    Une \important{fonction} $f: E \mapsto F$ où $E, F \subset \mathbb{R}$ est une application (une règle) qui donne pour tout élément $x \in D\left(F\right) = E$, un élément $y = f\left(x\right) \in F$. La notion réellement importante c'est que, à un élément de $D\left(f\right)$ il ne peut être associé qu'une seule valeur de $f\left(D\right)$.


    On dit que $D\left(f\right)$ est le \important{domaine de définition} de $f$ (qui est égal à $E$). De plus, on note $f\left(D\right)$ \important{l'ensemble image} (qui est un sous-ensemble $F$).

    On note $x \mapsto f\left(x\right)$.
}

\parag{Définition de graphique}{
    Le \important{graphique} de $f: E \mapsto F$ est l'ensemble des points sur le plan $\mathbb{R}^2$ avec les coordonnées $\left(x, f\left(x\right)\right)$.
}

\parag{Fonctions données par formule}{
    Si la fonction est donnée par une formule, alors $D\left(f\right)$ est le plus grand sous-ensemble de $\mathbb{R}$ où l'expression $f\left(x\right)$ est bien définie.
}

\parag{Croissance}{
    $f\left(x\right)$ est \important{croissante} sur $D\left(f\right)$ si $\forall x_1 < x_2 \in D\left(f\right)$, alors on a:
    \[f\left(x_1\right) \leq f\left(x_2\right)\]

    $f\left(x\right)$ est \important{strictement croissante} sur $D\left(f\right)$ si, pour les mêmes hypothèses, alors: 
    \[f\left(x_1\right) < f\left(x_2\right)\]
    
    On note $f\left(x\right)\uparrow$ dans les deux cas.
}

\parag{Décroissance}{
    $f\left(x\right)$ est \important{décroissante} sur $D\left(f\right)$ si $\forall x_1 < x_2 \in D\left(f\right)$, alors on a:
    \[f\left(x_1\right) \geq f\left(x_2\right)\]

    $f\left(x\right)$ est \important{strictement décroissante} sur $D\left(f\right)$ si, pour les mêmes hypothèses, alors: 
    \[f\left(x_1\right) > f\left(x_2\right)\]
    
    On note $f\left(x\right)\downarrow$ dans les deux cas.
}

\parag{Monotonicité}{
    Si $f$ est croissante ou décroissante sur $D\left(f\right)$, alors elle est \important{monotone}.

    De la même manière, $f$ est \important{strictement monotone} si elle est strictement croissante ou strictement décroissante sur son domaine de définition.
}

\parag{Parité}{
    On dit qu'un domaine de définition est symétrique si: 
    \[x \in D\left(f\right) \implies -x \in D\left(f\right)\]
    

    $f$ est \important{paire} si $D\left(f\right)$ est symétrique, et si $f$ est telle que 
    \[f\left(-x\right) = f\left(x\right), \mathspace \forall x \in D\left(f\right)\]

        
    $f$ est \important{impaire} si $D\left(f\right)$ est symétrique et si elle est telle que:
    \[f\left(-x\right) = -f\left(x\right), \mathspace \forall x \in D\left(f\right)\]
}

\parag{Périodicité}{
$f: E \mapsto F$ est \important{périodique} s'il existe $P \in \mathbb{R}^*$ tel que pour tout $x \in E$, alors $x + P \in E$ et: 
 \[f\left(x + P\right) = f\left(x\right), \mathspace \forall x \in E\]

 On appelle $P$ une \important{période} de $f$. On remarque que si $P$ est une période, alors $nP$ avec $n \in \mathbb{N}$ est aussi une période. Elle n'est donc pas unique.

 Puisque, si $f$ est périodique alors $x \in E \implies x + nP \in E$, on en déduit que $E$ n'est pas borné.

 Il est souvent (mais pas toujours ; par exemple : les fonctions constantes) possible de trouver \important{la plus petite période}, $P > 0$, telle que $\left\{nP\right\}_{n \in \mathbb{Z}}$ continent l'ensemble des périodes de la fonction périodique. (Pour que cet ensemble contienne toutes les périodes de la fonction, il faut bel et bien que $P$ soit la plus petite (puisque $n$ est dans $\mathbb{Z}$)).
}

\parag{Exemple 1}{
    Disons que nous voulons trouver la périodicité de la fonction suivante:
    \[f\left(x\right) = \sin^2\left(x\right), \mathspace x \in \mathbb{R}\]

    On veut écrire $\sin^2\left(x\right)$ sous la forme d'une somme de $\sin$ et $\cos$ (puisqu'on connait leur période, qui est $2\pi$ par définition (puisqu'elles paramétrisent un cercle de rayon $1$)):
    \[\cos\left(2x\right) = \cos\left(x + x\right) = \cos^2\left(x\right) - \sin^2\left(x\right) = 1 - \sin^2\left(x\right) - \sin^2\left(x\right) = 1 - 2\sin^2\left(x\right)\]

    On en déduit donc: 
    \[\sin^2\left(x\right) = \frac{1}{2}\left(1 - \cos\left(2x\right)\right)\]
    
    Puisque multiplier $x$ correspond à diviser la période, on trouve que la plus petite période est $P = \pi$.
}

\parag{Exemple 2}{
    Prenons la fonction suivante:
    \begin{functionbypart}{f\left(x\right)}
    0, \mathspace x \in \mathbb{Q}  \\
    1, \mathspace x \not\in \mathbb{Q}
    \end{functionbypart}

    On a donc que $D\left(f\right) = \mathbb{R}$. On sait qu'un nombre rationnel plus un nombre rationnel donne un nombre rationnel. De la même manière, on peut démontrer par la contraposée qu'un nombre rationnel plus un nombre irrationnel donne un nombre irrationnel. 

    On en déduit que $P = 1$ est une période, mais aussi que n'importe quel $P \in \mathbb{Q}$ est une période de $f$. En d'autres mots, $f$ est périodique mais elle n'admet pas de plus petite période. 

}

\parag{Bornes}{
    On dit que $f : E\mapsto F$ est \important{majorée} sur $A \subset E$ si l'ensemble $f\left(A\right) \subset \mathbb{R}$ est majoré.

    On dit que $f$ est \important{minorée} sur $A \subset E$ si l'ensemble $f\left(A\right) \subset \mathbb{R}$ est minorée.

    Si $f\left(x\right)$ est à la fois minorée et majorée sur $A$, alors elle est \important{bornée} sur $A$. C'est équivalent à la propriété suivante:
    \[\exists M \in \mathbb{R}_+ \telque \left|f\left(x\right)\right|_{x \in A} \leq M\]
    
    On définit \important{la borne supérieure} de $f$: 
    \[\sup_{x \in A} f\left(x\right) \over{=}{déf} \sup\left\{f\left(x\right), x \in A\right\}\]

    De la même manière, la \important{borne inférieure} est donnée par: 
    \[\inf_{x \in A} f\left(x\right) \over{=}{déf}  \inf\left\{f\left(x\right), x \in A\right\}\]
}

\parag{Exemple}{
    Soit $f\left(x\right) = x^2 + 3$ sur $A = \left]0,1\right[ $. On voit bien que $f\left(x\right)$ est bornée. On a donc: 
    \[\sup_{x \in A} f\left(x\right) = \sup\left\{x^2 + 3, x \in \left]0,1\right[ \right\} = 4\]
    
    De la même manière : 
    \[\inf_{x \in A} f\left(x\right) = \inf\left\{x^2 + 3, x \in \left]0,1\right[ \right\} = 3\]
    
}

\parag{Maximum et minimum locaux}{
    Soit $f: E \mapsto F$, avec $x_{0} \in E$. On dit que $f$ admet un \important{maximum local} au point $x_0$ s'il existe $\delta > 0$ tel que pour tout $x \in D\left(f\right)$, alors:
    \[\left|x - x_0\right| \leq \delta \implies f\left(x\right) \leq f\left(x_0\right)\]
    
    En d'autre mots, il existe un voisinage autour de $x_0$, dans lequel toutes les images sont plus grandes ou égales à celle de $f\left(x_0\right)$.

    \imagehere[0.5]{IllustrationMaximumLocal.png}

    De la même manière, on dit que $f$ admet un \important{minimum local} au point $x_0$ s'il existe $\delta > 0$ tel que pour tout $x \in D\left(f\right)$, alors:
    \[\left|x - x_0\right| \leq \delta \implies f\left(x\right) \geq f\left(x_0\right)\]
}

\parag{Maximum et minimum locaux}{
    Soit $f: E \mapsto F$ et $M \in \left\{f\left(x\right), x \in E\right\} = f\left(E\right)$ (c'est juste une manière \textit{fancy} de dire que $M$ est une valeur de la fonction), tels que pour tout $x \in E$, alors on a 
    \[f\left(x\right) \leq M\]
    
    On appelle $M$ le \important{maximum global} de $f$, et on le note: 
    \[\max_{x \in E} f\left(x\right) = M\]
    
    
    De la même manière, si on a $m \in f\left(E\right)$, tel que pour tout $x \in E$, alors on a 
    \[f\left(x\right) \geq m\]
    
    On appelle $m$ le \important{minimum global} de $f$, et on le note:
    \[\min_{x \in E} f\left(x\right) = m\]

    Si $f\left(x_0\right) = M$ ou $f\left(x_0\right) = m$, on dit que la fonction $f$ atteint son maximum global, ou son minimum global, respectivement, sur $E$ au point $x_0$.
}

\parag{Remarque 1}{
    Si $\max_{x \in E} f\left(x\right)$ existe, alors $f$ est majorée sur $E$, et 
    \[\sup_{x \in E} f\left(x\right) = \max_{x \in E} f\left(x\right)\]

    De la même manière, si $\min_{x \in E} f\left(x\right)$ existe, alors $f$ est minorée sur $E$, et 
    \[\inf_{x \in E} f\left(x\right) = \min_{x \in E} f\left(x\right)\]
}   

\parag{Remarque 2}{
    Une fonction bornée sur $E$ n'atteint pas forcément son min ou max sur $E$. 

    \subparag{Exemple}{
        Par exemple, $f\left(x\right) = x^2 + 3$ sur $E = \left]0,1\right[ $ est bornée, mais elle n'atteint ni son minimum, ni son maximum sur $E$.
    }
}

\parag{Surjectivité}{
    Une fonction $f: E \mapsto F$ est \important{surjective} si pour tout $y \in F$, il existe au moins un $x \in E$ tel que $f\left(x\right) = y$.

    \subparag{Remarque}{
        Si $f$ n'est pas surjective, on peut réduire l'ensemble d'arrivée $F$ pour que cela devienne le cas.
    }
}

\parag{Injectivité}{
    Une fonction $f: E \mapsto F$ est \important{injective} si pour tout $y \in F$, il existe au plus un $x \in E$ tel que $f\left(x\right) = y$.

    La définition suivante est équivalente:
    \[f\left(x_1\right) = f\left(x_2\right) \implies x_1 = x_2\]
    avec $x_1, x_2 \in E$.

    \subparag{Remarque}{
        Si $f$ n'est pas injective, alors on peut réduire l'ensemble de départ $E$ pour que cela devienne le cas.
    }

    \subparag{Test sur un graphique}{
        Si on peut tracer une droite horizontale qui croise plus d'une fois la fonction, alors elle n'est pas injective.
    }
}

\parag{Bijectivité}{
    Si une fonction $f: E \mapsto F$ est à la fois injective et surjective, alors elle est aussi \important{bijective}. 
}

\parag{Réciprocité}{
    Si $f: E \mapsto F$ est bijective, on peut définir la fonction réciproque par la formules suivante: 
    \[y = f\left(x\right), x \in E \implies x = f^{-1}\left(y\right), y \in F\]
}

\parag{Exemple}{
    La fonction suivante n'est pas injective
    \[\begin{split}
        f: \mathbb{R} &\longmapsto \left[-1, 1\right]  \\
        x &\longmapsto \cos\left(x\right)
    \end{split}\]

    Cependant, elle est bijective sur $\left[0, \pi\right] $.

    Par convention, on choisit les domaines suivants pour que les fonctions trigonométriques soient bijectives:
    \begin{itemize}
        \item $\sin x : \left[-\frac{\pi}{2}, \frac{\pi}{2}\right] \mapsto \left[-1, 1\right] $
        \item $\cos x : \left[0, \pi\right] \mapsto \left[-1, 1\right] $
        \item $\tan x : \left]-\frac{\pi}{2}, \frac{\pi}{2}\right[ \mapsto \mathbb{R}$
        \item $\cot x : \left]0, \pi\right[ \mapsto \mathbb{R}$
    \end{itemize}

    On peut donc définir les fonctions réciproques:
    \begin{itemize}
        \item $\arcsin x : \left[-1, 1\right] \mapsto \left[-\frac{\pi}{2}, \frac{\pi}{2}\right]$
        \item $\arccos x : \left[-1, 1\right] \mapsto \left[0, \pi\right]$
        \item $\arctan x : \mathbb{R} \mapsto \left]-\frac{\pi}{2}, \frac{\pi}{2}\right[$
        \item $\arccot x : \mathbb{R} \mapsto \left]0, \pi\right[$
    \end{itemize}
}

\parag{Remarque}{
    Les graphiques des fonctions réciproques sont symétriques par rapport à la droite $y = x$.
}

\parag{Exemple (courant en examen)}{
    Soit la fonction suivante: 
    \[f\left(x\right) = 2\sin\left(1 - x^2\right)\]
    
    On veut trouver un plus grand ensemble où $f\left(x\right)$ est bijective, et donner une fonction réciproque et son domaine de définition. On peut résoudre l'inéquation suivante:
    \[\frac{-\pi}{2} \leq 1 - x^2 \leq \frac{-\pi}{2} \implies -1 - \frac{\pi}{2} \leq -x^2 \leq -1 + \frac{\pi}{2} \implies 1 - \frac{\pi}{2} \leq x^2 \leq 1 + \frac{\pi}{2}\]

    De là, puisque $x^2 \geq 0$, on en déduit que:
    \[0 \leq x^2 \leq 1 + \frac{\pi}{2} \implies -\sqrt{1 + \frac{\pi}{2}} \leq x \leq \sqrt{1 + \frac{\pi}{2}}\]

    Cependant, puisqu'il y a un $x^2$, il nous faut encore couper cet intervalle ($x^2$ n'est pas bijective sur cet ensemble). On a deux moitiés, aucune n'est ``meilleure'' que l'autre, donc on peut faire un choix arbitraire. Par exemple, on peut choisir: 
    \[0 \leq x \leq \sqrt{1 + \frac{\pi}{2}} \implies D\left(f\right) = \left[0, \sqrt{1 + \frac{\pi}{2}}\right]\]
    
    Il faudrait justifier qu'on soit parti de $1 - x^2 \in \left[\frac{-\pi}{2}, \frac{\pi}{2}\right]$ et non pas, $1 - x^2 \in \left[\frac{\pi}{2}, \frac{3\pi}{2}\right]$, par exemple. On peut le justifier en disant que $1 - x^2$ est de plus en plus petit (ouai bon je suis pas fan de cet argument, ce que j'aurais fait ça aurait été de résoudre l'inéquation $\frac{n\pi}{2} \leq 1 - x^2 \leq \frac{n\pi}{2} + \pi$, puis de trouver le meilleur $n$).
    
    On peut maintenant chercher la fonction inverse: 
    \[2\sin\left(1 - x^2\right) = y \implies \sin\left(1 - x^2\right) = \frac{y}{2} \implies 1 - x^2 = \arcsin\left(\frac{y}{2}\right) \implies x^2 = 1 - \arcsin\left(\frac{y}{2}\right)\]

    Donc, puisque $0 \leq x \leq \sqrt{1 + \frac{\pi}{2}}$, $x$ est positif, donc on trouve que: 
    \[x = \sqrt{1 - \arcsin\left(\frac{y}{2}\right)} \implies f^{-1}\left(x\right) = \sqrt{1 - \arcsin\left(\frac{x}{2}\right)}\]
    
    Pour trouver son domaine de définition, $D\left(f^{-1}\right) = f\left(D\right)$ on peut simplement regarder les bornes de $f$, puisque cette fonction est croissante sur cet intervalle. On a: 
    \[f\left(0\right) = 2\sin\left(1\right) \mathspace \text{et} \mathspace f\left(\sqrt{1 + \frac{\pi}{2}}\right) = -2\]
    
    On a donc: 
    \[f: \left[0, \sqrt{1 + \frac{\pi}{2}}\right] \mapsto \left[-2, 2\sin\left(1\right)\right] \implies D\left(f^{-1}\right) = \left[-2, 2\sin\left(1\right)\right] \]
    
    
}


\end{document}
