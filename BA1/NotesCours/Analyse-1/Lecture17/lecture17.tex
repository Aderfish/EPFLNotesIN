\documentclass[a4paper]{article}

% Expanded on 2021-11-22 at 10:16:16.

\usepackage{../../style}

\title{Analyse 1}
\author{Joachim Favre}
\date{Lundi 22 novembre 2021}

\begin{document}
\maketitle

\lecture{17}{2021-11-22}{Des calculs de différences, c'est facile non?}{
\begin{itemize}
    \item Explication de 3 autres corollaires du TVI.
    \item Définition de fonction dérivable.
    \item Définition de fonction différentiable, et de fonction dérivée.
    \item Définition des dérivées sur le côté.
    \item Démonstration qu'une fonction dérivable est continue.
    \item Explication des opérations algébriques sur les dérivées, et de la dérivée de la composée de deux fonctions dérivables.
\end{itemize}

}


\parag{Corollaire 2}{
    Soient $I$ un intervalle ouvert et $f : I \mapsto \mathbb{R}$ une fonction continue strictement monotone. Alors, $f\left(I\right)$ est un intervalle ouvert

    \imagehere{TVICorollaire2.png}

    \subparag{Étude des hypothèses}{
        Si la fonction n'est pas continue, alors $f\left(I\right)$ pourrait être l'union d'un intervalle ouvert et d'un intervalle semi-ouvert (voir image du milieu).


        Si la fonction n'est pas strictement monotone, alors elle pourrait atteindre son maximum (par exemple), et donc $f\left(I\right)$ serait un intervalle semi-ouvert (voir image de droite (rire de droite)).
    }
}

\parag{Corollaire 3}{
    Toute fonction injective continue sur un intervalle est strictement monotone. 

    \imagehere{TVICorollaire3.png}

    \subparag{Étude des hypothèses}{
        Si la fonction est continue mais pas injective, alors elle n'est pas forcément monotone (voir image du milieu).

        On ne sait rien aussi si la fonction est injective mais pas continue (voir image de droite).
    }
}

\parag{Corollaire 4}{
    Toute fonction bijective continue sur un intervalle admet une fonction réciproque qui est continue et strictement monotone.

    \subparag{Remarque}{
        Le ``strictement monotone'' découle directement du fait que la fonction réciproque soit aussi bijective, donc on peut utiliser le corollaire 3.

        Le point important est que si la fonction $f$ est continue, alors $f^{-1}$ est continue.
    }
}

\section{Calcul différentiel}
\subsection{Fonctions dérivables}
\parag{Définition de fonction dérivable}{
    Une fonction $f: E \mapsto F$ est dite \important{dérivable} en $x_0 \in E$ si la limite suivante existe: 
    \[\lim_{x \to x_0} \frac{f\left(x\right) - f\left(x_0\right)}{x - x_0} \over{=}{déf} f'\left(x_0\right) \in \mathbb{R}\]
    
    Cette limite est appelée la \important{dérivée} de $f$ en $x_0$, notée $f'\left(x_0\right)$.


    \subparag{Note personnelle}{
        En faisant le changement de variable $h = x - x_0$, on a: 
        \[f'\left(x_0\right) = \lim_{x \to x_0} \frac{f\left(x\right) - f\left(x_0\right)}{x - x_0} = \lim_{h \to 0} \frac{f\left(x_0 + h\right) - f\left(x_0\right)}{h}\]
        
        Cela peut être plus pratique pour les calculs.
    }
    
}

\parag{Exemple 1}{
    Soient la fonction $f\left(x\right) = x^2$ et $x_0 \in \mathbb{R}$:
    \[\lim_{x \to x_0} \frac{x^2 - x_0^2}{x - x_0} = \lim_{x \to x_0} \frac{\left(x - x_0\right)\left(x + x_0\right)}{x - x_0} = \lim_{x \to x_0} \left(x + x_0\right) = 2x_0\]

    On obtient donc que: 
    \[\left(x^2\right)' = 2x, \mathspace \forall x \in \mathbb{R}\]
}

\parag{Exemple 2}{
    Soient $f\left(x\right) = \cos\left(x\right)$ et $x_0 \in \mathbb{R}$. On a: 
    \begin{multiequality}
    \lim_{x \to x_0} \frac{\cos\left(x\right) - \cos\left(x_0\right)}{x - x_0} & = \lim_{x \to x_0} \frac{\cos\left(\frac{x + x_0}{2} + \frac{x - x_0}{2}\right) - \cos\left(\frac{x + x_0}{2} - \frac{x - x_0}{2}\right)}{x - x_0}  \\
    & = \lim_{x \to x_0} \frac{-2\sin\left(\frac{x + x_0}{2}\right)\sin\left(\frac{x - x_0}{2}\right)}{x - x_0}  \\
    & = \lim_{x \to x_0} -\underbrace{\frac{\sin\left(\frac{x - x_0}{2}\right)}{\frac{x - x_0}{2}}}_{\to 1} \underbrace{\sin\left(\frac{x + x_0}{2}\right)}_{\to \sin\left(x_0\right)}  \\
    & = -\sin\left(x_0\right)  
    \end{multiequality}

    On utilise la continuité de $\sin$ pour calculer cette limite.

    On trouve donc que: 
    \[\left(\cos x\right)' = -\sin x, \mathspace \forall x \in \mathbb{R} \]

    \subparag{Remarque personnelle}{
        Je trouve personnellement qu'utiliser le changement de variable $h = x - x_0$ simplifie nos calculs: 
        \begin{multiequality}
        & \lim_{h \to 0} \frac{\cos\left(x_0 + h\right) - \cos\left(x_0\right)}{h} \\
        = & \lim_{h \to 0} \frac{\cos\left(x_0\right)\cos\left(h\right) - \sin\left(x_0\right)\sin\left(h\right) - \cos\left(x_0\right)}{h}  \\
        = & \lim_{h \to 0} \left(\cos\left(x_0\right) \underbrace{\frac{\cos\left(h\right) - 1}{h}}_{\to 0} - \sin\left(x_0\right) \underbrace{\frac{\sin\left(h\right)}{h}}_{\to 1}\right)  \\
        = & -\sin\left(x\right) 
        \end{multiequality}
         
    }
    
}

\parag{Exercice au lecteur}{
    Démontrer que:
    \begin{itemize}
        \item $\left(x^3\right)' = 3x^2, \mathspace \forall x \in \mathbb{R}$
        \item $\left(\sin x\right)' = \cos x, \mathspace \forall x \in \mathbb{R}$
    \end{itemize}
}

\parag{Fonction différentiable}{
    Si $f$ est dérivable en $x = x_0$, on peut écrire:
    \[f\left(x\right) = f\left(x_0\right) + f'\left(x_0\right)\left(x - x_0\right) + r\left(x\right), \mathspace \text{où } r\left(x\right) \over{=}{déf} f\left(x\right) - f\left(x_0\right) - f'\left(x_0\right)\left(x - x_0\right)\]

    On remarque que: 
    \[\lim_{x \to x_0} \frac{r\left(x\right)}{x - x_0} = \lim_{x \to x_0}  \left(\underbrace{\frac{f\left(x\right) - f\left(x_0\right)}{x - x_0}}_{\to f'\left(x_0\right)} - f'\left(x_0\right)\right) = 0\]
    
    Ainsi, on en déduit que toute fonction dérivable en $x = x_0$ admet une présentation: 
    \[f\left(x\right) = f\left(x_0\right) + a\left(x - x_0\right) + r\left(x\right), \mathspace \text{où } \lim_{x \to x_0} \frac{r\left(x\right)}{x - x_0} = 0\]
    
    Dans ce cas, on dit que $f$ est \important{différentiable} en $x_0$.

    Réciproquement, si:
    \[f\left(x\right) = f\left(x_0\right) + a\left(x - x_0\right) + r\left(x\right), \mathspace \telque \lim_{x \to x_0} \frac{r\left(x\right)}{x - x_0} = 0\]
    
    Alors: 
    \[\lim_{x \to x_0} \frac{f\left(x\right) - f\left(x_0\right)}{x - x_0} = \lim_{x \to x_0} \underbrace{\frac{a\left(x - x_0\right)}{x - x_0}}_{= a} + \underbrace{\frac{r\left(x\right)}{x - x_0}}_{\to 0} = a\]

    Ainsi, on en déduit que $f$ est dérivable en $x_0$ si et seulement si elle est différentiable en $x = x_0$, et $f'\left(x\right) = a$.
}

\parag{Définition de fonction dérivée}{
    Soit $f : E \mapsto F$ une fonction dérivable sur un ensemble $D\left(f'\right) \subset E$.

    On définit la \important{fonction dérivée}:
    \[\begin{split}
        f': D\left(f'\right) &\longmapsto \mathbb{R} \\
        x &\longmapsto f'\left(x\right)
    \end{split}\]
}

\parag{Exemple}{
    Voici les fonctions dérivées qu'on connait:
    \begin{center}
    \begin{tabular}{c|c}
        $f\left(x\right)$ & $f'\left(x\right)$ \\
        \hline
        $\cos\left(x\right)$ & $-\sin\left(x\right)$ \\
        $\sin\left(x\right)$ & $\cos\left(x\right)$  \\
        $x^2$ & $2x$ \\
        $x^3$ & $3x^2$
    \end{tabular}
    \end{center}
}

\parag{Interprétation géométrique}{
    \imagehere[0.6]{InterpretationGeometriqueDerivee.png}

    On voit que $\frac{f\left(x\right) - f\left(x_0\right)}{x - x_0}$ est la pente de la droite passant par $\left<x_0, f\left(x_0\right)\right>$ et $\left<x, f\left(x\right)\right>$. Donc, quand $x \to x_0$, on obtient la pente de la tangente à la courbe $y = f\left(x\right)$ en $x = x_0$.

    L'équation de la tangente est donnée par: 
    \[y - f\left(x_0\right) = f'\left(x_0\right)\left(x - x_0\right) \iff y = f\left(x_0\right) + f'\left(x_0\right)\left(x - x_0\right)\]
}

\parag{Définition des dérivées sur le côté}{
    La \important{dérivée à droite} est définie par: 
    \[f'_d\left(x_0\right) \over{=}{déf}  \lim_{x \to x_0^+} \frac{f\left(x\right) - f\left(x_0\right)}{x - x_0}\]

    De la même manière, la \important{dérivée à gauche} est définie par:
    \[f'_{{\color{red}g}}\left(x_0\right) \over{=}{déf} \lim_{{\color{red}x \to x_0^-}} \frac{f\left(x\right) - f\left(x_0\right)}{x - x_0} \]

    On remarque que $f'\left(x_0\right)$ existe si et seulement s'il existe $f'_d\left(x_0\right)$ et $f'_g\left(x_0\right)$, et que $f'_d\left(x_0\right) = f'_g\left(x_0\right)$
}

\parag{Proposition}{
    Une fonction dérivable en $x = x_0$ est continue en $x = x_0$.

    \subparag{Preuve}{
        On voit que: 
        \begin{multiequality}
        \lim_{x \to x_0} f\left(x\right) & = \lim_{x \to x_0} \left(f\left(x_0\right) + f\left(x\right) - f\left(x_0\right)\right)  \\
        & = \lim_{x \to x_0} \left(f\left(x_0\right) + \underbrace{\frac{f\left(x\right) - f\left(x_0\right)}{x - x_0}}_{\to f'\left(x_0\right)}\underbrace{\left(x - x_0\right)}_{\to 0}\right)  \\
        & = f\left(x_0\right) 
        \end{multiequality}

        Pour calculer cette limite, on utilise que $f$ est dérivable en $x_0$.
        
        \qed
    }
    
    \subparag{Réciproque}{
        La réciproque de cette proposition est fausse; une fonction continue n'est pas nécessairement dérivable. 

        Par exemple, $f\left(x\right) = \left|x\right|$ est continue en $x = 0$:
        \[\lim_{x \to 0^+} \left|x\right| = \lim_{x \to 0^+} x = 0, \mathspace \lim_{x \to 0^-} \left|x\right| = \lim_{x \to 0^-} \left(-x\right) = 0\]

        On peut donc en déduire que $\left|x\right|$ est continue en $x = 0$: 
        \[\lim_{x \to 0} \left|x\right| = 0 = f\left(0\right)\]
        
        
        Calculons maintenant les dérivées sur le côté: 
        \[f'_d\left(0\right) = \lim_{x \to 0^+} \frac{\left|x\right|}{x} = \lim_{x \to 0^+} \frac{x}{x} = 1\]
        \[f'_g\left(0\right) = \lim_{x \to 0^-} \frac{\left|x\right|}{x} = \lim_{x \to 0^-} \frac{-x}{x} = -1\]
        
        Puisque $f'_d\left(0\right) \neq f'_g\left(0\right)$, on en déduit que $f'\left(0\right)$ n'existe pas.

        Graphiquement, on voit qu'on peut bien dessiner $\left|x\right|$ sans lever le crayons, donc elle est continue, cependant elle n'est pas ``lisse'' en $x = 0$, il y a un angle, donc elle n'y est pas dérivable.
    }
}

\parag{Remarque}{
    On peut introduire la limite infinie si: 
    \[\lim_{x \to x_0} \frac{f\left(x\right) - f\left(x_0\right)}{x - x_0} = \pm \infty\]

    Puisque la limite n'existe pas, $f$ n'est pas dérivable en $x = x_0$. 

    Dans ce cas, le graphique de $f$ admet une tangente verticale en $x_0$.
}

\parag{Exemple}{
    Prenons la fonction suivante:
    \begin{functionbypart}{f\left(x\right)}
    \sqrt[3]{x}, \mathspace x \geq 0 \\
    -\sqrt[3]{\left|x\right|}, \mathspace x < 0
    \end{functionbypart}

    On a besoin de définir notre fonction comme ça car on a uniquement défini les racines $n$-èmes pour les $x$ positifs. Calculons les dérivées sur le côté:
    \[f'_d\left(0\right) = \lim_{x \to 0^+} \frac{\sqrt[3]{x} - 0}{x - 0} = \lim_{x \to 0^+} \frac{x^{\frac{1}{3}}}{x} = \lim_{x \to 0^+} \frac{1}{x^{\frac{2}{3}}} = +\infty\]
    \[f'_g\left(0\right) = \lim_{x \to 0^-} \frac{\sqrt[3]{x} - 0}{x - 0} = \lim_{x \to 0^-} \frac{-\sqrt[3]{\left|x\right|}}{x} \over{=}{$y = -x$} \lim_{y \to 0^+} \frac{-\sqrt{y}}{-y} = \lim_{y \to 0^+} \frac{1}{y^{\frac{2}{3}}} = +\infty\]
    
    On en déduit que $f'\left(0\right) = +\infty$. On peut bien voir que, graphiquement, il y a une tangente verticale en $x = 0$.

    \imagehere[0.4]{TangenteVerticale.png}
}

\parag{Opérations algébriques sur les dérivées}{
    Soient $f, g : E \mapsto F$ deux fonctions dérivables en $x = x_0$. Alors:
    \begin{enumerate}
        \item $\left(\alpha f + \beta g\right)'\left(x_0\right) = \alpha f'\left(x_0\right) + \beta g'\left(x_0\right), \mathspace \forall \alpha, \beta \in \mathbb{R}$
        \item $\left(f\cdot g\right)'\left(x_0\right) = f'\left(x_0\right)g\left(x_0\right) + f\left(x_0\right)g'\left(x_0\right)$
        \item $\left(\dfrac{f}{g}\right)'\left(x_0\right) = \dfrac{f'\left(x_0\right)g\left(x_0\right) - g'\left(x_0\right)f\left(x_0\right)}{g^2\left(x\right)}$, si $g\left(x\right) \neq 0$ au voisinage de $x_0$
    \end{enumerate}
    
    \subparag{Preuve du point (3)}{
        Nous voulons montrer que, si $g\left(x_0\right) \neq 0$:
        \[\left(\frac{1}{g}\right)'\left(x_0\right) = -\frac{g'\left(x_0\right)}{g^2\left(x_0\right)}\]

        On pourra alors utiliser ce fait avec la propriété (2) pour démontrer la propriété (3).

        On a: 
        \begin{multiequality}
        \left(\frac{1}{g}\right)'\left(x_0\right) & = \lim_{x \to x_0} \frac{\frac{1}{g\left(x\right)} - \frac{1}{g\left(x_0\right)}}{x - x_0}  \\
        & = \lim_{x \to x_0} \frac{g\left(x_0\right) - g\left(x\right)}{g\left(x\right)g\left(x_0\right)\left(x - x_0\right)}  \\
        & = \lim_{x \to x_0} \underbrace{\frac{g\left(x_0\right) - g\left(x\right)}{x - x_0}}_{\to -g'\left(x_0\right)} \cdot \underbrace{\frac{1}{g\left(x\right)g\left(x_0\right)}}_{\to \frac{1}{g^2\left(x_0\right)}}  \\
        & = - \frac{g'\left(x_0\right)}{g^2\left(x_0\right)} 
        \end{multiequality}
        
        On utilise que $g$ est dérivable et que $g$ est continue.
    }
}

\parag{Exemple}{
    Soit $f\left(x\right) = \tan\left(x\right) = \frac{\sin x}{\cos x}$. Par la propriété (3), on a: 
    \[\left(\tan\left(x\right)\right)' = \frac{\left(\sin\left(x\right)\right)' \cos\left(x\right) - \sin\left(x\right)\left(\cos\left(x\right)\right)'}{\cos^2\left(x\right)} = \frac{\cos^2\left(x\right) + \sin^2\left(x\right)}{\cos^2\left(x\right)} = \frac{1}{\cos^2\left(x\right)}\]

    On en déduit donc que: 
    \[\left(\tan\left(x\right)\right)' = \frac{1}{\cos^2\left(x\right)}\]

    \subparag{Remarque personnelle}{
        On remarque que: 
        \[1 + \tan^2\left(x\right) = \frac{\cos^2\left(x\right) + \sin^2\left(x\right)}{\cos^2\left(x\right)} = \frac{1}{\cos^2\left(x\right)}\]

        Ces résultats seront très importants quand nous calculerons des intégrales.
    }
}

\parag{Proposition}{
    Soit $f\left(x\right) = x^{n}$, où $n \in \mathbb{N}^*$. On a:
    \[\left(x^{n}\right)' = nx^{n - 1}, \mathspace \forall x \in \mathbb{R}\]
    
    \subparag{Preuve}{
        On va démontrer ce fait par récurrence.

        \important{Initialisation:} Prenons $n = 1$. Alors: 
        \[\left(x\right)' = \lim_{x \to x_0} \frac{x - x_0}{x - x_0} = 1 \implies \left(x\right)' = 1 \mathspace \forall x \in \mathbb{R}\]
        
        \important{Hérédité:} Supposons que $\left(x^n\right)' = nx^{n-1}$ pour un $n \in \mathbb{N}^*$. On veut montrer que $\left(x^{n+1}\right)' = \left(n+1\right)x^n$:
        \[\left(x^{n+1}\right)' = \left(x^n \cdot x\right)' = \left(x^n\right)' x + x^n \left(x\right)' = nx^{n-1} x + x^{n}\cdot 1\]

        On peut donc en déduire que:
        \[\left(x^{n+1}\right)' = nx^{n} + x^n = \left(n+1\right)x^{n}\]

        Ainsi, par récurrence, la proposition $\left(x^n\right)' = nx^{n-1}$ est vraie pour tout $n \in \mathbb{N}^*$.

        \qed
    }
}

\parag{Proposition (dérivée de la fonction composée de deux fonctions dérivables)}{
    Soient $f : E\mapsto F$ une fonction dérivable en $x = x_0 \in E$ et $g : G \mapsto H$ dérivable en $f\left(x_0\right)$, où $f\left(E\right) \subset G$.

    Nous avons: 
    \[\left(g \circ f\right)'\left(x_0\right) = g'\left(f\left(x_0\right)\right)f'\left(x_0\right)\]
    
    \subparag{Preuve}{
        Nous voulons calculer la limite suivante: 
        \[\lim_{x \to x_0} \frac{g\left(f\left(x\right)\right) - g\left(f\left(x_0\right)\right)}{x - x_0}\]
        
        Si $f\left(x\right) \neq f\left(x_0\right)$ au voisinage de $x_0$, on peut prendre: 
        \[\lim_{x \to x_0} \underbrace{\frac{g\left(f\left(x\right)\right) - g\left(f\left(x_0\right)\right)}{f\left(x\right) - f\left(x_0\right)}}_{\to g'\left(f\left(x_0\right)\right)} \cdot \underbrace{\frac{f\left(x\right) - f\left(x_0\right)}{x - x_0}}_{\to f'\left(x\right)} = f'\left(f\left(x_0\right)\right)f'\left(x_0\right)\]

        Si $f\left(x\right) = f\left(x_0\right)$, alors les deux côtés sont nul: 
        \[\frac{g\left(f\left(x_0\right)\right) - g\left(f\left(x_0\right)\right)}{x - x_0} = 0 = g'\left(f\left(x_0\right)\right) \cdot \underbrace{\frac{f\left(x\right) - f\left(x_0\right)}{x - x_0}}_{= 0}\]
        
        Donc, dans les deux cas, on a que cela tend vers $g'\left(f\left(x_0\right)\right)f'\left(x_0\right)$. Ainsi: 
        \[\left(g \circ f\right)'\left(x_0\right) = g'\left(f\left(x_0\right)\right)f'\left(x_0\right)\]
        
        \qed
    }
}

\parag{Exemple}{
    Soit la fonction suivante: 
    \[g\left(x\right) = \frac{\sin^2\left(x\right) + \cos^4\left(x\right)}{1 + \sin^2\left(x\right) + \cos^4\left(x\right)}\]
    
    Prenons le changement de variable $f\left(x\right) = \sin^2\left(x\right) + \cos^4\left(x\right)$: 
    \[g\left(x\right) = \frac{f\left(x\right)}{1 + f\left(x\right)} = \frac{1 + f\left(x\right) - 1}{1 + f\left(x\right)} = 1 - \frac{1}{1 + f\left(x\right)}\]
    
    Ainsi, on peut maintenant calculer la dérivée de $g$:
    \[g'\left(x\right) = \left(1 - \frac{1}{1 + f\left(x\right)}\right)' = -\left(\frac{1}{1 + f\left(x\right)}\right)' = -\frac{-f'\left(x\right)}{\left(1 + f\left(x\right)\right)^2} = \frac{f'\left(x\right)}{\left(1 + f\left(x\right)\right)^2}\]
    
    Nous voulons maintenant calculer la dérivée de $f$. Avant, nous pouvons voir le fait suivant: 
    \[\left(\sin^k\left(x\right)\right)' = k\left(\sin\left(x\right)\right)^{k-1} \cos\left(x\right), \mathspace \left(\cos^k\left(x\right)\right)' = k \left(\cos\left(x\right)\right)^{k-1} \left(-\sin\left(x\right)\right)\]
    
    Donc: 
    \begin{multiequality}
        f'\left(x\right) & = \left(\sin^2\left(x\right) + \cos^4\left(x\right)\right)' \\
        & = 2\sin\left(x\right)\cos\left(x\right) + 4\cos^3 \left(x\right)\left(-\sin\left(x\right)\right)   \\
        & = \sin\left(2x\right) - 2\sin\left(2x\right)\cos^2\left(x\right)  \\
        & = \sin\left(2x\right)\underbrace{\left(1 - 2 \cos^2\left(x\right)\right)}_{-\cos\left(2x\right)}  \\
        & = -\sin\left(2x\right)\cos\left(2x\right)  \\
        & = -\frac{\sin\left(4x\right)}{2} 
    \end{multiequality}
    
    On peut donc finir de calculer $g'$: 
    \[g'\left(x\right) = \frac{f'\left(x\right)}{\left(1 + f\left(x\right)\right)^2} = -\frac{\sin\left(4x\right)}{2\left(1 + \sin^2\left(x\right) + \cos^4\left(x\right)\right)^2}\]
    
}





\end{document}
