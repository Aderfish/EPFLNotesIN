\documentclass{article}

% Expanded on 2021-10-04 at 10:05:37.

\usepackage{../../style}

\title{Analyse 1}
\author{Joachim Favre}
\date{Lundi 04 octobre 2021}

\begin{document}
\maketitle

\lecture{4}{2021-10-04}{Début des nombres complexes}{
\begin{itemize}[left=0pt]
    \item Définition des nombres complexes, et de leur addition, multiplication, soustraction et division.
    \item Définition de l'exponentielle complexe.
    \item Définition des trois formes des nombres complexes (cartésienne, polaire trigonométrique et polaire exponentielle).
    \item Démonstration de la formule de De Moivre.
\end{itemize}

}

\section{Nombres complexes}
\parag{Limitation de $\mathbb{R}$}{
    On a vu que l'équation $x^2 = 2$ n'a pas de solution dans $\mathbb{Q}$. Maintenant, on remarque que \important{l'équation $x^2 = -1$ n'a pas de solution dans $\mathbb{R}$}.

    On introduit donc le symbole \important{$i$}, tel que $i^2 = -1$.

    \subparag{Exercice au lecteur}{
        Démontrer que cette équation n'a effectivement pas de solution dans les réels. Il faut démontrer en utilisant les axiomes de $\mathbb{R}$ que:
        \begin{enumerate}
            \item Si $x > 0$, alors $-x\leq 0$, $\forall x \in \mathbb{R}$
            \item $\forall x \in \mathbb{R}$, $0\cdot x = 0$
            \item $\forall x \in \mathbb{R}$, $x\cdot x = \left(-x\right)\cdot\left(-x\right) \geq 0$ en utilisant (1) et (2).
            \item $-1 < 0$
        \end{enumerate}

        Par (3) et (4) on en déduit que $x^2 = -1$ n'a pas de solution.
    }

    \subparag{Correction}{
        \begin{enumerate}[left=0pt]
            \item Partons de notre hypothèse:
                \[x \geq 0 \implies x + \left(-x\right) \geq -x \implies 0 \geq \left(-x\right)\]
            \item Partons de notre côté gauche:
                \[0\cdot x = \left(0 + 0\right)\cdot x = 0\cdot x + 0\cdot x\]

            En simplifiant par $0\cdot x$ des deux côtés (en ajoutant l'opposé aux deux côtés, pour être plus formel), on obtient que:
             \[0 = 0\cdot x\]
         \item On remarque que:
             \[0 = 0\cdot x = \left(x + \left(-x\right)\right)\cdot x = x\cdot x + \left(-x\right)\cdot x\]
          Donc, en additionnant l'opposé de $x\cdot x$ aux deux côtés, c'est équivalent à:
          \[-\left(x\cdot x\right) = \left(-x\right)\cdot x \over{=}{\text{commutativité}} x\cdot\left(-x\right)\]

          De la même manière,
          \[0 = 0\left(-x\right) = \left(x + \left(-x\right)\right)\left(-x\right) = x\left(-x\right) + \left(-x\right)\left(-x\right)=0 \]

          Qui implique que:
          \[-\left(\left(-x\right)\left(-x\right)\right) = x\left(-x\right) = \left(-x\right)x\]

          Puisqu'on obtiens deux résultats égaux ($\left(-x\right)x = \left(-x\right)x$), on sait donc que
          \[-\left(x\cdot x\right) = -\left(\left(-x\right)\left(-x\right)\right) \implies x\cdot x = \left(-x\right)\left(-x\right)\]

          Ensuite, par un des axiomes de la relation d'ordre de $\mathbb{R}$, on sait que
          \[x \geq 0 \implies \underbrace{x\cdot x}_{= \left(-x\right)\left(-x\right)} \geq 0 \]

          Et, de manière similaire:
          \[x \leq 0 \over{\implies}{(1)} \left(-x\right) \geq 0 \implies \underbrace{\left(-x\right)\left(-x\right)}_{= x\cdot x} \geq 0\]

          \item Par l'axiome de l'existence de l'identité multiplicative, on sait que $1 \neq 0$, ce qui implique que $-1 \neq 0$. Maintenant, si on suppose par l'absurde que $-1 > 0$, on a que :
              \[-1 > 0 \implies \left(-1\right)\left(-1\right) = 1\cdot 1= 1 > 0\]

              Ce qui est une contradiction par le point (1), un nombre et son opposé (ici $1$ et $-1$) ne peuvent pas être les deux positifs.

              \qed
        \end{enumerate}
    }
}

\parag{Nombres complexes}{
    On considère les expressions de la forme $\left\{z = a + ib\right\} = \mathbb{C}$, où $a, b \in \mathbb{R}$, avec les opérations suivantes:
    \begin{itemize}
        \item[\important{+} :] $\left(a + ib\right) + \left(c + id\right) = \left(a + c\right) + i\left(b + d\right)$, qui ressemble énormément à la manière dont on additionne des vecteurs de $\mathbb{R}^2$. Par les axiomes de $\mathbb{R}$, on en déduit qu'elle est associative et commutative. De plus, il existe un élément neutre: $0 + 0i = 0$, il existe un opposé par rapport à l'addition: $\left(-a + i\left(-b\right)\right) + \left(a + ib\right) = 0 + i0 = 0 \in \mathbb{C}$.
        \item[\important{$\cdot$ :}] $\left(a + ib\right) \cdot \left(c + id\right) = ac - bd + i\left(ad - bc\right)$ qui est aussi associative et commutative. De plus, il existe aussi un élément neutre: $\left(a + ib\right)\left(1 + i0\right) = a + ib$ et il existe un réciproque si $z \in \mathbb{C}$, $z \neq 0$, qu'on appelle $z^{-1}$ et qui est tel que $z\cdot z^{-1} = z^{-1} \cdot z = 1$.
    \end{itemize}

    \subparag{Note personnelle}{
        La définition de la multiplication entre deux nombres complexes est cohérente si on utilise simplement la distributivité et le fait que $i^2 = -1$:
        \[\left(a + ib\right)\left(c + id\right) = ac + adi + bci + bdi^2 = ac - bd + i\left(ad + bc\right)\]
    }
}

\parag{Proposition pour le nombre réciproque}{
    Soit $z = a + ib \neq 0$. Alors,
    \[z^{-1} = \frac{a - ib}{a^2 + b^2}\]

    On remarque que, puisque $z \neq 0$, alors on peut bien diviser par $a^2 + b^2$.

    \subparag{Vérification}{
        \[z\cdot z^{-1} = \left(a + ib\right) \frac{a - ib}{a^2 + b^2} = \frac{a^2 + b^2 + i\left(ab - ba\right)}{a^2 + b^2} = \frac{a^2 + b^2}{a^2 + b^2} = 1\]

        \qed
    }

    \subparag{Note personnelle}{
        La définition de l'inverse peut être retrouvée facilement, si on multiplie notre fraction ``par 1, mais pas n'importe quel 1'' (comme disait ma prof de maths du collège, Ms. Lemmon). On utilise simplement une des identités remarquables, comme lorsqu'on souhaite rationaliser un dénominateur:
        \[\frac{1}{a + ib} = \frac{1}{a + ib} \cdot \frac{a - ib}{a - ib} = \frac{a - ib}{a^2 - \left(ib\right)^2} = \frac{a - ib}{a^2 + b^2}\]
    }
}

\parag{Distributivité}{
    Le produit est distributif sur l'addition:
    \[z_1\left(z_2 + z_3\right) = z_1 z_2 + z_1 z_3 \mathspace \forall z_1, z_2, z_3 \in \mathbb{C}\]
    \subparag{Preuve}{
        Laissée au lecteur comme exercice.
    }


    Grâce à toutes ces propriétés, on sait que $\mathbb{C}$ est un corps commutatif.
}


\parag{L'ordre de $\mathbb{C}$}{
    On remarque que $\mathbb{C}$ n'est pas ordonné.

    \subparag{Preuve}{
        Si $i > 0$, alors $i^2 = -1 > 0$, ce qui est une contradiction.
        Si $i < 0$, alors $\left(-i\right)^2 = -1 > 0$, ce qui est aussi une contradiction.

        On avait bien démontré que $-1 < 0$ dans $\mathbb{R} \subset \mathbb{C}$, donc $\mathbb{C}$ ne peut pas être ordonné.
        \qed
    }

    De plus, on a deux racines complexes de l'équation $z^2 = -1$:
    \[z_1 = i \mathspace \text{et} \mathspace z_2 = -i\]
}

\subsection{Les trois formes des nombres complexes}

\parag{Forme cartésienne}{
    Les nombres sont la forme
    \[z = a + ib, \mathspace a,b \in \mathbb{R}\]
    sont sous \important{forme cartésienne}.

    On note aussi $z = \text{Re}\left(z\right) + \text{Im}\left(z\right) = \Re\left(z\right) + i\Im\left(z\right)$, où $\Re\left(z\right) = a\in \mathbb{R}$ dénomme la partie réelle de $z$, et $\Im\left(z\right) = b \in \mathbb{R}$ est la partie imaginaire de $z$. Attention, la partie imaginaire n'est pas $ib$, mais bien $b$.
}

\parag{Définition du module}{
    On appelle le \important{module} de $z$:
    \[\left|z\right| \over{=}{déf} \sqrt{\left(\Re\left(z\right)\right)^2 + \left(\Im\left(z\right)\right)^2} = \sqrt{a^2 + b^2} \geq 0\]

    On remarque que $\left|z\right| = 0 \iff z = 0$.

    \imagehere{FormePolaireNombresComplexes.png}
}

\parag{Forme polaire (trigonométrique)}{
    On remarque qu'on peut aussi paramétriser notre vecteur selon sa longueur, $\rho$, et son angle $\phi$:
    \[z = \rho\left(\cos\left(\phi\right) + i\sin\left(\phi\right)\right) \mathspace \rho \geq 0, \phi \in \mathbb{R}\]

    Où $\Re\left(z\right) = \rho \cos \phi$ et $\Im\left(z\right) = \rho \sin\left(\phi\right)$. De plus,
    \[\left|z\right| = \sqrt{\rho^2 \cos^2 \phi + \rho^2 \sin^2 \phi} = \rho \geq 0\]

    Donc, le module de $z$, c'est $\rho$. De plus, \important{l'argument de $z$} ($z \neq 0$, sinon ça ne fait aucun sens), $\phi$, est défini à $2\pi k$, $k \in \mathbb{Z}$. Donc,
    \[\rho \neq 0 \implies \sin \phi = \frac{\Im\left(z\right)}{\rho} \mathspace \cos \phi = \frac{\Re\left(z\right)}{\rho}\]

    Ou alors, on peut diviser l'un par l'autre et on obtient
    \[\tan \phi = \frac{\Im\left(z\right)}{\Re\left(z\right)} = \frac{b}{a}\]

    si $a = \Re\left(z\right) \neq 0$, qui est un problème puisqu'on veut une formule générale (y compris pour $a = 0$).
}

\parag{À la recherche de l'argument}{
    On veut trouver une formule qui nous permette de trouver l'argument de n'importe quel nombre complexe $z = a + ib \neq 0$.

    On va utiliser la fonction $\arctan\left(x\right)$. Or, on sait qu'elle est définie de $\mathbb{R}$ vers $\left]-\frac{\pi}{2}, \frac{\pi}{2}\right[ $.

    Séparons notre formule pour différents cas:
    \begin{itemize}
        \item \important{Si $z = a + ib$, $a > 0$}: alors c'est facile:
            \[\phi = \arg\left(z\right) \in \left]-\frac{\pi}{2}, \frac{\pi}{2}\right[ \implies \arg\left(z\right) = \arctan\left(\frac{b}{a}\right) \in \left] -\frac{\pi}{2}, \frac{\pi}{2}\right[ \]
            à $2k\pi$ près, $k \in \mathbb{Z}$.

        \item \important{Si $z = a + ib$, $a < 0$}: $\arctan\left(\frac{b}{a}\right)$ est à un $\pi$ près, donc on peut prendre
            \[\phi = \arg\left(z\right) \in \left] \frac{\pi}{2}, \frac{3\pi}{2}\right[ \implies \arg\left(z\right) = \arctan\left(\frac{b}{a}\right) + \pi\]
            à $2k\pi$ près, $k \in \mathbb{Z}$.
        \item \important{Si $\Re\left(z\right) = a = 0$}: Alors, on a deux cas.
            \[\arg\left(z\right) = \frac{\pi}{2}, \text{ si } \Im\left(z\right) = b > 0 \text{ ou } \arg\left(z\right) = \frac{3\pi}{2}, \text{ si } \Im\left(z\right) = b < 0\]

    \end{itemize}

    Pour résumer:
    \begin{functionbypart}{\arg\left(a + bi\right)}
        \arctan\left(\frac{b}{a}\right), \mathspace \text{si } a > 0 \\
        \arctan\left(\frac{b}{a}\right) + \pi, \mathspace \text{si } a < 0 \\
        \frac{\pi}{2}, \mathspace \text{si } a = 0 \text{ et } b > 0 \\
        \frac{3\pi}{2}, \mathspace \text{si } a = 0 \text{ et } b < 0
    \end{functionbypart}

    \subparag{Note personnelle}{
        En informatique, on appelle cette fonction \texttt{arctan2(b,a)}, et elle existe dans beaucoup de librairies standard (oui c'est un anglicisme, mais vous auriez préféré ``bibliothèque logicielle standard'' ?) puisqu'elle est utile à chaque fois qu'on a besoin de connaître l'angle d'un point sur un cercle quelconque à partir de ses coordonnées.
    }

    \subparag{Remarque}{
        Il est important de réaliser que l'argument de $z$ est seulement défini pour les nombres complexes non nuls.
    }
}

\parag{Forme polaire exponentielle}{
    Soit $z = x + iy \in \mathbb{C}$. Alors
    \[e^{z} = \underbrace{\exp\left(z\right)}_{\text{exp complexe}} \over{=}{déf} \underbrace{e^x}_{\text{exp réelle}} \left(\cos\left(y\right) + i\sin\left(y\right)\right)\]

    \subparag{Pourquoi est-ce une bonne définition?}{
        Si $z = x \in \mathbb{R} \subset \mathbb{C} \implies e^z = e^x\left(\cos\left(0\right) + i\sin\left(0\right)\right) = e^x$.

        Si $z = iy \in i\mathbb{R} \subset \mathbb{C} \implies e^z = \cos\left(y\right) + i\sin\left(y\right)$, qui est la formule d'Euler.

        De plus, on a

        \begin{multiequality}
        e^{iy_1} \cdot e^{i y_2} & = \left(\cos\left(y_1\right) + i\sin\left(y_1\right)\right)\left(\cos\left(y_2\right) + i\sin\left(y_2\right)\right) \\
        & = \cos y_1 \cos y_2 - \sin y_1 \sin y_2 + i\left(\sin y_1 \cos y_2 + \cos y_1 \sin y_2\right) \\
        & = \cos\left(y_1 + y_2\right) + i\sin\left(y_1 + y_2\right) \\
        & = e^{i\left(y_1 + y_2\right)}
        \end{multiequality}



        À cela s'ajoute que, si $y_1 = y_2 + 2k\pi$, alors:
        \[e^{i y_1} = e^{i\left(y_2 + 2k\pi\right)} = \cos\left(y_2 + 2k\pi\right) + i\sin\left(y_2 + 2k\pi\right) = \cos\left(y_2\right) + i\sin\left(y_2\right) = e^{i y_2}\]

        Pour résumer,
        \[e^{i y_1} e^{i y_2} = e^{i\left(y_1 + y_2\right)} \text{ et } e^{i y_1} = e^{i y_2} \text{ si } y_1 - y_2 = 2k\pi, k \in \mathbb{Z}\]
    }

    \subparag{Non, mais vraiment, pourquoi c'est une bonne définition ? (Note personnelle)}{
        Il est très difficile d'aborder le sujet de l'exponentielle complexe auprès d'étudiants, j'en ai aucun doute ; cette méthode permet de ne rien avoir à prouver de très spécial, mais n'est pas la plus intuitive, à mon sens. Vous êtes certainement encore en train de vous demander d'où sort le cercle ($x = \cos\left(t\right)$ et $y = \sin\left(t\right)$ est la paramétrisation d'un cercle avec un angle $t$, donc cela veut dire que $e^{it}$ trace un cercle sur le plan complexe).

        Je vais faire une affirmation ici, pour l'instant il va falloir me croire sur parole (enfin c'est du texte, pas de la parole, mais bon), on peut démontrer que la règle de la dérivée en chaine fonctionne aussi pour les fonctions complexes. De là, je vous invite à aller regarder la très bonne vidéo de 3Blue1Brown sur l'intuition derrière la formule d'Euler: \url{https://www.youtube.com/watch?v=v0YEaeIClKY} (y'a des sous-titres dans beaucoup de langues si vous êtes allergique à l'anglais).

        Maintenant que vous avez cette intuition, il faut bien définir l'exponentielle, et partir directement de la formule d'Euler, comme ce qu'on a fait dans le cadre de ce cours, peut toujours paraître ``dangereux'' (c'est bien beau d'avoir une intuition, mais c'est pas très formel). Cependant, on peut bien démontrer cette formule, en utilisant l'affirmation que j'ai faite ci-dessus par rapport à la règle de la dérivée en chaine. En effet, si on prend la fonction suivante:
        \[\begin{split}
        f: \mathbb{R} &\longmapsto \mathbb{C} \\
        x &\longmapsto \frac{e^{ix}}{\cos\left(x\right) + i\sin\left(x\right)}
        \end{split}\]

        On peut démontrer que $f'\left(x\right) = 0$ pour tout $x \in \mathbb{R}$, donc que la fonction est constante. Ceci implique que $f\left(x\right) = f\left(0\right)$ pour tout $x$ (on aurait pu prendre n'importe quel constante, mais $f\left(0\right)$ est pratique). On remarque que $f\left(0\right) = 1$, et de là il est facile de montrer que l'égalité d'Euler tient. (Je tiens à préciser que cette preuve ne vient pas de moi, mais qu'on la vue l'année passée en cours (big up André-Chavanne)).

        Maintenant, pour atteindre notre définition vue ci-dessus, il nous manque juste un petit bout: on sait pas comment $e^{x + yi}$ se comporte. Cependant, là on peut justement la définir avec une propriété qui, elle, semble logique puisqu'elle marche pour les nombres dans $\mathbb{R}$:
        \[e^{x + y} = e^{x} e^{y} \mathspace \forall x, y \in \mathbb{C}\]

        Et de là on a bien que:
        \[e^{x + yi} = e^{x} e^{yi} = e^{x} \left(\cos\left(y\right) + i\sin\left(y\right)\right) \mathspace \forall x, y \in \mathbb{R}\]
        et, je l'espère, c'est plus intuitif.

        Toute cette réflexion nous aura permis de définir notre exponentielle complexe en définissant uniquement le comportement de $e^{x + y}$ avec $x, y \in \mathbb{C}$, ce qui est donc une supposition beaucoup moins grande que celle faite en classe, sauf qu'il faut utiliser de l'analyse complexe, qu'on ne verra pas en Analyse 1.
    }
}

\parag{Formule d'Euler}{
    Une autre formule qui porte le nom d'Euler est que
    \[e^{i\pi} = \cos\pi + i\sin\pi = -1 \implies e^{i\pi} + 1 = 0\]

    Qui est une formule très intéressante, puisqu'elle relie trois constantes fondamentales, $e$, $i$ et $\pi$; le 0 et le 1; ainsi que l'addition, la multiplication et la puissance.
}

\parag{Les trois formes}{
    On a donc
    \[z = \overbrace{\Re\left(z\right) + i\Im\left(z\right)}^{\text{cartésienne}} = \overbrace{\left|z\right|\left(\cos\left(\arg\left(z\right)\right) + i\sin\left(\arg\left(z\right)\right)\right)}^{\text{polaire trigonométrique}} = \overbrace{\left|z\right|e^{i\arg\left(z\right)}}^{\text{polaire exponentielle}}\]
    où, pour passer entre les deux formes polaires c'est très facile, mais il faut faire des calculs pour passer à la forme cartésienne.

    De plus, on a le module de $z$:
    \[\left|z\right| = \sqrt{\Re\left(z\right)^2 + \Im\left(z\right)^2}\]

    Et pour l'argument de $z = a + bi$:
    \begin{functionbypart}{\arg\left(a + bi\right)}
        \arctan\left(\frac{b}{a}\right), \mathspace \text{si } a > 0 \\
        \arctan\left(\frac{b}{a}\right) + \pi, \mathspace \text{si } a < 0 \\
        \frac{\pi}{2}, \mathspace \text{si } a = 0 \text{ et } b > 0 \\
        \frac{3\pi}{2}, \mathspace \text{si } a = 0 \text{ et } b < 0
    \end{functionbypart}

    \subparag{Remarque}{
        La forme polaire peut être notée sous la forme:
        \[z = \rho\left(\cos\left(\phi\right) + i\sin\left(\phi\right)\right) = \rho e^{i\phi}\]
        où $\rho > 0$ et $\phi \in \mathbb{R}$.
    }
}

\parag{Exemple 1}{
    Soit $z = 1 - i$. Quelle est sa forme polaire? Le module se trouve facilement:
    \[\left|z\right| = \sqrt{1^2 + ^2} = \sqrt{2}\]

    De plus, pour l'argument, puisque $\Re\left(z\right) = 1 > 0$, on a:
    \[\arg\left(z\right) = \arctan\left(\frac{\Im\left(z\right)}{\Re\left(z\right)}\right) = \arctan\left(\frac{-1}{1}\right) = \arctan\left(-1\right) = -\frac{\pi}{4}\]

    Donc,
    \[z = 1 - i = \sqrt{2}\left(\cos\left(-\frac{\pi}{4}\right) + i\sin\left(-\frac{\pi}{4}\right)\right) = \sqrt{2}e^{-i \frac{\pi}{4}}\]

}

\parag{Exemple 2}{
    Si on a $z = 3 e^{i \frac{5\pi}{6}}$. Quelle est sa forme cartésienne?
    \[z = 3 e^{i \frac{5\pi}{6}} = 3\left(\cos\left( \frac{5\pi}{6}\right) + i\sin\left( \frac{5\pi}{6}\right)\right) = 3\left( - \frac{\sqrt{3}}{2} + i \frac{1}{2}\right) = \frac{-3\sqrt{3}}{2} + i \frac{3}{2}\]
}


\subsection{Multiplication en forme polaire}

\parag{Proposition}{
    Soient $z_1 = \rho_1 e^{i \phi_1}$ et $z_2 = \rho_2 e^{i \phi_2}$ deux nombres complexes non-nuls. Alors,
    \[z_1 z_2 = \rho_1 \rho_2 e^{i\left(\phi_1 + \phi_2\right)}\]

    \subparag{Preuve}{
        On a
        \[z_1 z_2 = \rho_1 e^{i \phi_1} \rho_2 e^{i \phi_2} = \rho_1 \rho_2 e^{i\left(\phi_1 + \phi_2\right)}\]

        Notez qu'on a déjà démontré que $e^{i\phi_1}e^{i\phi_2} = e^{i\left(\phi_1 + \phi_2\right)}$ quand ou a voulu justifier l'exponentielle complexe (pas dans ma note personnelle).
    }

}

\parag{Rotation}{
    Si $z_1 = \left|z_1\right| e^{i\arg\left(z_1\right)}$ et $z_2 = \left|z_2\right| e^{i\arg\left(z_2\right)}$, alors
    \[z_1 z_2 = \left|z_1\right| \left|z_2\right| e^{i\left(\arg z_1 + \arg z_2\right)}\]

    Donc, multiplier deux nombres complexes revient à multiplier leur ``longueur'' et additionner leur angle.

    \subparag{Note personnelle}{
        Cette propriété est très pratique en informatique lorsque nous voulons faire tourner quelque chose. C'est, entre autre, une des propriétés qui a fait que les mathématiciens ont développé les quaternions (équivalent des nombres complexes en quatre dimensions) ; ils sont pratiques pour les rotations en 3D (et 4D, mais bon).
    }
}

\parag{Exemple 3}{
    Soit $z = e^{i\phi}$ ($\left|z\right| = 1$) et $w = e^{i \theta}$ ($\left|w\right| = 1$). Alors,
    \[zw = e^{i\left(\phi + \theta\right)}\]
}

\parag{Division en forme polaire}{
    Si on a $z = \left|z\right| e^{i\arg z} = \rho e^{i \phi}$, avec $z \neq 0$. Alors:
    \[z^{-1} = \frac{1}{\rho} e^{-i \phi}\]

    \subparag{Preuve}{
        On a bien:
        \[z z^{-1} = \rho e^{i \phi} \cdot \frac{1}{\rho} e^{-i\phi} = \frac{\rho}{\rho} e^{i\left(\phi - \phi\right)} = 1 e^{i0} = 1\]
    }

    De plus, de manière générale, si $z_1 = \left|z_1\right| e^{i \arg z_1}$ et $z_2 = \left|z_2\right| e^{i \arg z_2}$, avec $z_1 \neq 0$ et $z_2 \neq 0$:
    \[\frac{z_1}{z_2} = \frac{\left|z_1\right|}{\left|z_2\right|} e^{i\left(\arg z_1 - \arg z_2\right)}\]
}

\parag{Proposition (Formule de De Moivre)}{
    Pour tout $\rho > 0$, $\phi \in \mathbb{R}$ et $n \in \mathbb{N}^*$, on a:
    \[\left(\rho\left(\cos\left(\phi\right) + i\sin\left(\phi\right)\right)\right)^n = \rho^n \left(\cos\left(n \phi\right) + i\sin\left(n \phi\right)\right)\]

   En forme exponentielle, c'est équivalent à dire que:
   \[\left(\rho e^{i \phi}\right)^n = \rho^n e^{in\phi}\]

   \subparag{Définition des démonstrations par récurrence}{
       \important{(0)} Proposition qui dépend de $n \in \mathbb{N}$ (avec $n \geq m$, $m \in \mathbb{N}$ fixé).

       \important{Intialisation: (1)} Démontrer que la proposition est vraie pour $n = m$.

       \important{Hérédité: (2)} Démontrer que si la proposition est vraie pour $n = k \in \mathbb{N}$, avec $k \geq m$, alors elle est vraie pour $n = k + 1$.

       Alors, la proposition est vraie pour
       \[n = m \over{\implies}{\text{(1)}}  n = m + 1 \over{\implies}{\text{(2)}}  n = m + 2 \over{\implies}{\text{(2)}}  \ldots\]

       On obtient donc bien tout $n \geq m$ par l'axiome de récurrence de $\mathbb{N}$.
   }

    \subparag{Démonstration}{
        \important{Initialisation: } Si on a $n = 1$, alors $\left(\rho e^{i\phi}\right)^1 = \rho e^{i1\phi}$, qui est vrai.

        \important{Hérédité: } Supposons que la proposition est vraie pour $n = k \in \mathbb{N}^*$. Démontrons que la proposition est vraie pour $n = k + 1$:
        \[\left(\rho e^{i\phi}\right)^{k+1} = \left(\rho e^{i\phi}\right)^k \left(\rho e^{i\phi}\right) \over{=}{\text{par la supposition}} \rho^k e^{ik\phi} \cdot \rho e^{i\phi}\]

        Or, c'est égal à:
        \[\rho^{k+1} e^{i\left(k\phi + \phi\right)} = \rho^{k+1} e^{i\left(k+1\right)\phi}\]

        Ce qui démontre notre proposition pour $n = k+1$.

        Par récurrence, la proposition est donc vraie pour tout $n \in \mathbb{N}^*$.

        \qed
    }


    \subparag{Remarque (note personnelle)}{
       Attention, ce théorème ne marche que pour $n \in \mathbb{Z}$. Sinon:
       \[1 = 1^{\frac{1}{4}} = \left(e^{2\pi i}\right)^{\frac{1}{4}} \over{=}{De Moivre} e^{\frac{\pi}{2}i} = i\]
       ce qui est une contradiction (cela vient de l'ambiguïté de la définition de la racine n-ième de $x$, puisqu'on a bien que $i^{4} = 1$).
   }

}

\end{document}
